\documentclass[12pt]{article}
\usepackage{amssymb}
\usepackage{amsmath}
\usepackage[usenames,dvipsnames]{xcolor}
\usepackage{geometry}
	\geometry{a4paper, total={170mm,257mm}, left=20mm, top=20mm, }
	
\renewcommand*\contentsname{Tartalomjegyzék}

\AddToHook{cmd/section/before}{\clearpage}


\begin{document}
\begin{titlepage}
	\centering \vfill
	{\textsc{Budapesti Műszaki és Gazdaságtudományi Egyetem} \par} \vspace{7cm}
	{\huge\bfseries A számítástudomány alapjai\par} \vspace{0.5cm}
	{\large \textsc{Összefoglaló jegyzet}\par} \vspace{1.5cm}
	{\Large\itshape Készítette: Illyés Dávid\par} \vfill

	\noindent\fbox{%
    	\parbox{140mm}{
			Ez  a jegyzet nagyon hasonlóan van struktúrálva az előadás jegyzetekhez és fő célja, hogy olyan módon adja át a "A Programozás Alapjai 1" nevű tárgy anyagát, hogy az teljesen kezdők számára is könnyen megérthető és megtanulható legyen. 
   		}
	}
	
	\vfill {\large \today\par}
\end{titlepage} 
\tableofcontents
\addtocontents{toc}{~\hfill\textbf{Oldal}\par}

	\section{A gráfelmélet alapjai}
		\subsection{Mi a gráf?}
			
			\textcolor{blue}{\textbf{Def:}} $G = (V,E)$ \textcolor{red}{egyszerű, irányítatlan gráf}
			
			\textcolor{purple}{\textbf{Példa:}} ha $ V \neq 0$ és $E \subseteq \binom{V}{2}$, ahol $\binom{V}{2} = \{\{u,v\} : u,v \in V, u \neq v\}$.
			
			$V$ a $G$ \textcolor{red}{csúcsainak} (vagy \textcolor{red}{(szög)pontjainak}), $E$ pedig G \textcolor{red}{éleinek} halmaza. 
			
			\textcolor{purple}{\textbf{Példa:}} $G = (\{a,b,c,d\},\{a,b\},\{a,c\},\{b,c\},\{b,d\})$
			
			\textcolor{blue}{\textbf{Def:}} A $G = (V,E)$ gráf \textcolor{red}{diagramja} a $G$ egy olyan lerajzolása, amiben $V$-nek a sík különböző pontjai felelnek meg, és $G$ minden $\{u,v\}$ élének egy $u$-t és $v$-t összekötő görbe felel meg.
			
			\textbf{Terminológia \& konvenciók:} Gráf alatt rendszerint egyszerű, irányítatlan gráfot értünk. Ha $G$ egy gráf, akkor $V(G)$ a $G$ csúcshalmazát, $E(G)$ pedig $G$ élhalmazát jelöli, azaz $G = (V(G),E(G))$. Az $e = \{u,v\}$ élt röviden $uv$-vel jelöljük. 
			
			Ekkor $e$ az $u$ és $v$ csúcsokat \textcolor{red}{köti össze}. Továbbá $u$ és $v$ az $e$ \textcolor{red}{végpontjai}, amelyek az $e$ élre \textcolor{red}{illeszkednek}, és $e$ mentén \textcolor{red}{szomszédosak}.
		
		\subsection{Multigráfok és irányított gráfok}

			\textcolor{blue}{\textbf{Megj:}} Ha egy gráf nem egyszerű, akkor lehetnek \textcolor{red}{párhuzamos élei, hurokélei} vagy akár párhuzamos hurokélei is.
			
			\textcolor{blue}{\textbf{Def:}} Az \textcolor{red}{irányított gráf} olyan gráf, aminek minden éle irányított.

			\textcolor{blue}{\textbf{Def:}} $G = (V,E)$ \textcolor{red}{véges gráf}, ha $V$ és $E$ is véges halmazok.

			\textcolor{blue}{\textbf{Def:}} Az \textcolor{red}{n-pontú út, n-pontú kör}, ill. \textcolor{red}{n-pontú teljes gráf} jele rendre $P_n$, $C_n$, ill. $K_n$. ($P_1,P_2,P_3$ elfajulók.) \textcolor{orange}{\textbf{Megf:}} $K_1 = P_1, P_2=C_2, C_3=K_3$

			\textcolor{blue}{\textbf{Def:}} $c \in V(G)$ esetén a $v$-re illeszkedő élek száma a $v$ fokszáma. Jelölése $d_g(v)$ vagy $d(v)$, a hurokél kétszer számít. (Irányított gráf esetén $\delta(v)$ ill. $\rho(v)$ a $v$ \textcolor{red}{ki-} ill. \textcolor{red}{befokát} jelöli.)

			\textcolor{blue}{\textbf{Def:}} A $G$ gráf maximális ill. minimális fokszáma $\Delta(G)$ ill. $\delta(G)$. $G$ \textcolor{red}{reguláris}, ha minden csúcsának foka ugyanannyi: $\Delta(G)=\delta(G),G$ pedig \textcolor{red}{k-reguláris}, ha minden csúcsának pontosan $k$ a fokszáma.

			\textcolor{orange}{\textbf{Megf:}} Minden kör 2-reguláris, $K_n$ pedig $(n-1)$-reguláris.

		\subsection{Handshaking lemma}

			\textcolor{orange}{\textbf{Kézfogás-lemma (KFL):}} Ha $G = (V,E)$ véges, nem feltétlenül egyszerű gráf, akkor $\sum_{v \in V} d(v)=2|E|$, azaz a csúcsok fokszámösszege az élszám kétszerese.
		
			\textcolor{orange}{\textbf{Általánosított kézfogás-lemma:}} Tetsz. $G = (V,E)$ véges irányított gráfra $\sum_{v \in V} \delta (v) = \sum_{v \in V} \rho (V) = |E|$, azaz a csúcsok ki- és befokainak összege is az élszámot adja meg. 

			\textcolor{green}{\textbf{Biz:}} Az egyes csúcsokból kilépő éleket megszámolva $G$ minden irányított élét pontosan egyszer számoljuk meg. Ezért a kifokok összege az élszám. A belépő éleket leszámlálva hasonló igaz, ezért a befokok összege is az élszám.
			
			\textcolor{green}{\textbf{A KFL bizonyítása:}} Készítsükel a $G'$ digráfot úgy, hogy $G$ minden élét egy oda-vissza irányított élpárral helyettesítjük. Ekkor \[\sum_{v \in V} d_G(V) = \sum_{v \in V} \delta_{G'} (v) = |E(G')| = 2|E(G)| \]

			\textcolor{blue}{\textbf{Megj:}} Úgy is bizonyíthattuk volna az általánosított kéfogás-lemmát, hogy egyenként húzzuk be $G$-be az éleket. 0-elű (\textcolor{red}{üres})gráfokra a lemma triviális, és minden egyes él behúzása pontosan 1-gyel növeli az élszámot is és a ki/befokok összegét is.

			\textcolor{blue}{\textbf{Megj:}} Úgy is bizonyíthattuk volna a kézfogás-lemmát, hogy egyenként húzzuk be $G$-be az éleket. Üresgráfokra a lemma triviális, és minden egyes él behúzása pontosan 2-vel növeli a kétszeres élszámot és a csúcsok fokszámösszeget is.


		\subsection{Komplementer és izomorfia}

			\textcolor{blue}{\textbf{Def:}}	A $G$ \textbf{egyszerű} gráf \textcolor{red}{komplementere} $\overline{G} = (V,(G), \binom{v}{2} \backslash E(G))$.

			\textcolor{blue}{\textbf{Megj:}} $G$ és $\overline{G}$ csúcsai megegyeznek, és két csúcs pontosan akkor szomszédos $\overline{G}$-ben, ha nem szomszédosak $G$-ben.

			\textcolor{purple}{\textbf{Példa:}}


		\subsection{Gráfoperációk}
		\subsection{Háromféle elérhetőség, összefüggőség}
		\subsection{Gráfok összefüggősége a gyakorlatban}
		\subsection{Fák és erdők}
		\subsection{Fák további tulajdonságai}
		\subsection{Feszítőfák}
		
		
	\section{Minimális költségű feszítőfák}
		\subsection{Alapkörrendszer, alap vágás renszer}
		\subsection{Minimális költségű feszítőfa}
		\subsection{Minimális költségű feszítőfák struktúrája}
		\subsection{Az ötödik elem}
		\subsection{Mkkffák egy villamosmérnöki alkalmazása}
		
	\section{Gráfbejárások és legrövidebb utak}
		
	\section{Legrövidebb utak, DFS, PERT}
		
	\section{Euler-séták és Hamilton-körök}
	
	\section{Síkgráfok}
	
	\section{Lineáris egyenletrendszerek}
	
	\section{Az \(\mathbb{R}^n\) tér alaptulajdonságai}
	
	\section{Altér bázisa és dimenziója}
	
	\section{Négyzetes mátrix determinánsa}
	
	\section{Mátrixműveletek és lineáris leképezések}
	
	\section{Mátrix rangja és inverze}
	
	\section{Mátrixegyenletek}
	
\end{document}
