\documentclass[12pt]{article}
\usepackage{amssymb}
\usepackage{amsmath}
\usepackage[usenames,dvipsnames]{xcolor}
\usepackage{geometry}
	\geometry{a4paper, total={170mm,257mm}, left=20mm, top=20mm, }
	
\renewcommand*\contentsname{Tartalomjegyzék}

\AddToHook{cmd/section/before}{\clearpage}

\begin{document}
\begin{titlepage}
	\centering \vfill
	{\textsc{Budapesti Műszaki és Gazdaságtudományi Egyetem} \par} \vspace{7cm}
	{\huge\bfseries A számítástudomány alapjai\par} \vspace{0.5cm}
	{\large \textsc{Összefoglaló jegyzet}\par} \vspace{1.5cm}
	{\Large\itshape Készítette: Illyés Dávid\par} \vfill

	\noindent\fbox{%
    	\parbox{140mm}{
			\color{red}\textbf{Ez  a jegyzet nagyon hasonlóan van struktúrálva az előadás jegyzetekhez és fő célja, hogy olyan módon adja át a "A Programozás Alapjai 1" nevű tárgy anyagát, hogy az teljesen kezdők számára is könnyen megérthető és megtanulható legyen. }
   		}
	}
	
	\vfill {\large \today\par}
\end{titlepage} 
\tableofcontents
\addtocontents{toc}{~\hfill\textbf{Oldal}\par}

	\section{A gráfelmélet alapjai}
		\subsection{Mi a gráf?}
			
			\textcolor{blue}{\textbf{Def:}} $G = (V,E)$ \textcolor{red}{egyszerű, irányítatlan gráf}
			
			\textcolor{purple}{\textbf{Példa:}} Ha $ V \neq 0$ és $E \subseteq \binom{V}{2}$, ahol $\binom{V}{2} = \{\{u,v\} : u,v \in V, u \neq v\}$. $V$ a $G$ \textcolor{red}{csúcsainak} (vagy \textcolor{red}{(szög)pontjainak}), $E$ pedig G \textcolor{red}{éleinek} halmaza. 
			
			\textcolor{purple}{\textbf{Példa:}} $G = (\{a,b,c,d\},\{a,b\},\{a,c\},\{b,c\},\{b,d\})$
			
			\textcolor{blue}{\textbf{Def:}} A $G = (V,E)$ gráf \textcolor{red}{diagramja} a $G$ egy olyan lerajzolása, amiben $V$-nek a sík különböző pontjai felelnek meg, és $G$ minden $\{u,v\}$ élének egy $u$-t és $v$-t összekötő görbe felel meg.
			
			\textbf{Terminológia \& konvenciók:} Gráf alatt rendszerint egyszerű, irányítatlan gráfot értünk. Ha $G$ egy gráf, akkor $V(G)$ a $G$ csúcshalmazát, $E(G)$ pedig $G$ élhalmazát jelöli, azaz $G = (V(G),E(G))$. Az $e = \{u,v\}$ élt röviden $uv$-vel jelöljük. 
			
			Ekkor $e$ az $u$ és $v$ csúcsokat \textcolor{red}{köti össze}. Továbbá $u$ és $v$ az $e$ \textcolor{red}{végpontjai}, amelyek az $e$ élre \textcolor{red}{illeszkednek}, és $e$ mentén \textcolor{red}{szomszédosak}.
		
		\subsection{Multigráfok és irányított gráfok}

			\textcolor{blue}{\textbf{Megj:}} Ha egy gráf nem egyszerű, akkor lehetnek \textcolor{red}{párhuzamos élei, hurokélei} vagy akár párhuzamos hurokélei is.
			
			\textcolor{blue}{\textbf{Def:}} Az \textcolor{red}{irányított gráf} olyan gráf, aminek minden éle irányított.

			\textcolor{blue}{\textbf{Def:}} $G = (V,E)$ \textcolor{red}{véges gráf}, ha $V$ és $E$ is véges halmazok.

			\textcolor{blue}{\textbf{Def:}} Az \textcolor{red}{n-pontú út, n-pontú kör}, ill. \textcolor{red}{n-pontú teljes gráf} jele rendre $P_n$, $C_n$, ill. $K_n$. ($P_1,P_2,P_3$ elfajulók.) \textcolor{orange}{\textbf{Megf:}} $K_1 = P_1, P_2=C_2, C_3=K_3$

			\textcolor{blue}{\textbf{Def:}} $c \in V(G)$ esetén a $v$-re illeszkedő élek száma a $v$ fokszáma. Jelölése $d_g(v)$ vagy $d(v)$, a hurokél kétszer számít. (Irányított gráf esetén $\delta(v)$ ill. $\rho(v)$ a $v$ \textcolor{red}{ki-} ill. \textcolor{red}{befokát} jelöli.)

			\textcolor{blue}{\textbf{Def:}} A $G$ gráf maximális ill. minimális fokszáma $\Delta(G)$ ill. $\delta(G)$. $G$ \textcolor{red}{reguláris}, ha minden csúcsának foka ugyanannyi: $\Delta(G)=\delta(G),G$ pedig \textcolor{red}{k-reguláris}, ha minden csúcsának pontosan $k$ a fokszáma.

			\textcolor{orange}{\textbf{Megf:}} Minden kör 2-reguláris, $K_n$ pedig $(n-1)$-reguláris.

		\subsection{Handshaking lemma}

			\textcolor{orange}{\textbf{Kézfogás-lemma (KFL):}} Ha $G = (V,E)$ véges, nem feltétlenül egyszerű gráf, akkor $\sum_{v \in V} d(v)=2|E|$, azaz a csúcsok fokszámösszege az élszám kétszerese.
		
			\textcolor{orange}{\textbf{Általánosított kézfogás-lemma:}} Tetsz. $G = (V,E)$ véges irányított gráfra $\sum_{v \in V} \delta (v) = \sum_{v \in V} \rho (V) = |E|$, azaz a csúcsok ki- és befokainak összege is az élszámot adja meg. 

			\textcolor{green}{\textbf{Biz:}} Az egyes csúcsokból kilépő éleket megszámolva $G$ minden irányított élét pontosan egyszer számoljuk meg. Ezért a kifokok összege az élszám. A belépő éleket leszámlálva hasonló igaz, ezért a befokok összege is az élszám. \raggedright \textcolor{blue}{$\Box$} 
			
			\textcolor{green}{\textbf{A KFL bizonyítása:}} Készítsükel a $G'$ digráfot úgy, hogy $G$ minden élét egy oda-vissza irányított élpárral helyettesítjük. Ekkor \[\sum_{v \in V} d_G(V) = \sum_{v \in V} \delta_{G'} (v) = |E(G')| = 2|E(G)| \;\;\;\;\;\textcolor{blue}{\Box} \]

			\textcolor{blue}{\textbf{Megj:}} Úgy is bizonyíthattuk volna az általánosított kéfogás-lemmát, hogy egyenként húzzuk be $G$-be az éleket. 0-elű (\textcolor{red}{üres})gráfokra a lemma triviális, és minden egyes él behúzása pontosan 1-gyel növeli az élszámot is és a ki/befokok összegét is.

			\textcolor{blue}{\textbf{Megj:}} Úgy is bizonyíthattuk volna a kézfogás-lemmát, hogy egyenként húzzuk be $G$-be az éleket. Üresgráfokra a lemma triviális, és minden egyes él behúzása pontosan 2-vel növeli a kétszeres élszámot és a csúcsok fokszámösszeget is.

		\subsection{Komplementer és izomorfia}

			\textcolor{blue}{\textbf{Def:}}	A $G$ \textbf{egyszerű} gráf \textcolor{red}{komplementere} $\overline{G} = (V,(G), \binom{v}{2} \backslash E(G))$.

			\textcolor{blue}{\textbf{Megj:}} $G$ és $\overline{G}$ csúcsai megegyeznek, és két csúcs pontosan akkor szomszédos $\overline{G}$-ben, ha nem szomszédosak $G$-ben.

			\textcolor{purple}{\textbf{Példa:}}

			\textcolor{orange}{\textbf{Megf:}} Ha $G = (V,E)$ egyszerű gárf és a $|V(G)| = n$, akkor $d_G(v)+d_{\overline{G}}=n-1$ teljesül $G$ bármely $v$ csúcsra.

			\textcolor{green}{\textbf{Biz:}} A $K_n$ teljes frág minden éle a $G$ és $\overline{G}$ gráfok közül pontosan az egyikhez tartozik. Ezért $d_G (v) + d_{\overline{G}}(v)$ megyegyezik a $v$ csúcs $K_n$-beli fokszámával, ami $n-1$. \raggedright \textcolor{blue}{$\Box$} 

			\textcolor{blue}{\textbf{Def:}} A $G$ és $G'$ gráfok akkor \textcolor{red}{izomorfak}, ha mindekét gráf csúcsai őgy számozhatók meg az 1-től $n$-ig terjedő egész számokkal (alkalmas $n$ esetén), hogy $G$ bármely két $u,v$ csúcsa között pontosan annyi él fut $G$-ben, mint az $u$-nak és $v$-nek megfelelő sorszámú csúcsok között $G'$-ben. Jelölése: $G \cong G'$.
			
			\textcolor{purple}{\textbf{Példa:}}

			\textcolor{orange}{\textbf{Megf:}} Ha $G \cong G'$, akkor $G$ és $G'$ lényegében ugyanúgy néznek ki. Így például minden fokszám ugyanannyiszor lép fel $G$-ben mint $G'$-ben, ugyan annyi $C_{42}$ kör található $G$-ben, mint $G'$-ben, stb.

		\subsection{Gráfoperációk}

			\textcolor{blue}{\textbf{Def:}} \textcolor{red}{Éltörlés}, \textcolor{green}{csúcstörlés}, \textcolor{brown}{élhozzáadás}.

			\textcolor{blue}{\textbf{Def:}} \textcolor{red}{Feszítő részgráf:} éltörlésekkel kapható gráf.

			\textcolor{green}{\textbf{Feszített részgráf:}} csúcstörlésekkel kapható gráf.

			\textcolor{orange}{\textbf{Részgráf:}} él- és csúcstörlésekkel kapható gráf.

			\textcolor{purple}{\textbf{Példa:}} \textcolor{red}{$H_1$}, \textcolor{green}{$H_2$}, \textcolor{orange}{$H_3$}: a $G$ \textcolor{red}{feszítő}, \textcolor{green}{feszített}, \textcolor{orange}{jelzőnélküli} részgráfjai.
		
			\textcolor{orange}{\textbf{Megf:}} $H$ a $G$ részgráfja $\Longleftrightarrow$ $V(H) \subseteq V(G)$ és $E(H) \subseteq E(G)$.

			$H$ a $G$ feszítő részgráfja $\Longleftrightarrow$ $V(H) = V(G)$ és $E(H) \subseteq E(G)$.

			$H$ a $G$ feszített részgráfja $\Longleftrightarrow$ $V(H) \subseteq V(G)$ és $E(H)$ a $H$.

			\textcolor{blue}{\textbf{Megj:}} A gráf definíciója megengedi, hogy a gráf egyik részéből egyáltalán ne vezessen él a gráf maradék részébe, azaz a gráf egyik csúcsáből ne lehessen eleken kereszül eljutni a gráf egy másik csúcsába. Ez történik pl. az üresgráf (alias $\overline{K_n}$) esetén.

		\subsection{Háromféle elérhetőség, összefüggőség}

			\textcolor{blue}{\textbf{Def:}} Legyen $G = (V,E)$ (irányított vagy irányítatlan) gráf.

			\textcolor{red}{\textbf{Élsorozat:}} ($v_1,e_1,v_2,e_2,\dots,v_k,e_k,v_{k+1}$), ahol $e_i=v_i v_{i+1}\forall i$. (Tkp egyik csúcsból eljutunk egy másik csúcsba mindig élek mentén haladva.)

			\textcolor{green}{Séta:} olyan élsorozat, amelyikban nincsen ismétlődő él.

			\textcolor{brown}{Út:} olyan séta, amelyikben nincs ismétlődő csúcs.

			\textcolor{purple}{\textbf{Terminológia:}} Ha a kezdőpont $u$, a végpont $v$, akkor \textcolor{red}{$uv$-élsorozatról}, \textcolor{green}{$uv$-sétáról}, ill. \textcolor{brown}{$uv$-útról} beszélünk. Ha hangsúlyozni szeretnénk, hogy $u = v$, de a kezdő (és vég)pontot nem akarjuk megnevezni, akkor \textcolor{red}{zárt élsorozatról}, \textcolor{green}{körsétáról} ill. \textcolor{brown}{körről} beszélünk.

			\textcolor{orange}{\textbf{Megf:}} $G$-ben $\exists uv$-út $\Rightarrow G$-ben $\exists uv$-séta $\Rightarrow G$-ben $\exists uv$-élsorozat \raggedright \textcolor{blue}{$\Box$} 

			\textcolor{orange}{\textbf{Állítás:}} $G$-ben $\exists uv$-élsorozat $\Rightarrow G$-ben $\exists uv$-út \raggedright \textcolor{blue}{$\Box$} 

			\textcolor{blue}{\textbf{Def:}} $G$ ir.tatlan gráf $u$-ból $v$ \textcolor{red}{\textbf{elérhető}} (\textcolor{red}{\textbf{$u \sim v$}}), ha $\exists uv$-út G-ben.

			\textcolor{blue}{\textbf{Def:}} A $G$ irányítatlan gráf \textcolor{red}{\textbf{összefüggő}}, ha $u \sim v \forall u,v \in V(G)$.

			\textcolor{blue}{\textbf{Megj:}} (1) Az összefüggőség szokásos definíciója nem a $\sim$ reláció segítségével történik, hanem valahogy így: a $G$ irányítatlan gráfot akkor mondjuk öszefüggőnek, ha $G$ bármely két csúcsa között vezet út $G$-ben.

			\textcolor{blue}{\textbf{Megj:}} (2) Az előző definíciót irányított fráfokra is kterjeszthető: a $G$ irányított gráfot akkor mondjuk \textcolor{red}{\textbf{erősen összefüggő}}nek, ha $G$ bármely $u,v \in V(G) $ esetén van \textbf{irányított} $uv$-út $G$-ben.

			\textcolor{blue}{\textbf{Megj:}} (3) Irányított gráf másfajta összefüggősége is értelmezhető: a $G$ irányított gráfot akkor mondjuk \textcolor{red}{\textbf{gyengén összefügő}}nek, ha a $G$-nek megfelelő irányítatlan gráf összefüggő.

			\textcolor{orange}{\textbf{Köv:}} Ha $G$ irányítatlan gráf, akkor $\sim$ ekvivalenciareláció:

			(1) $\forall u \in V(G) : u \sim u$, (2) $ \forall u,v \in V(G) : u \sim v \Rightarrow v \sim u$, és (3) $\forall u,v,w \in V(G): u \sim v \sim \ w \Rightarrow u \sim w$. \raggedright \textcolor{blue}{$\Box$} 

			\textcolor{blue}{\textbf{Def:}} A $G$ gráf \textcolor{red}{\textbf{(összefüggő) komponense}} a $\sim$ ekvivalenciaosztálya. Az egyelemű komponens neve \textcolor{red}{\textbf{izolált pont}}.

		\subsection{Gráfok összefüggősége a gyakorlatban}

			\textcolor{orange}{\textbf{Lemma:}} (1) $K \subseteq V(G)$ pontosan akkor komponense $G$-nek, ha $K$-ból nem lép ki éle $G$-nek, de $\forall v,v' \in$ esetén $v \sim v'$.

			(2) Minden $G$ irányítatlan gráf csúcshalmaza egyértelműen bomlik fel $G$ komponenseinek diszjunkt uniójára. \raggedright \textcolor{blue}{$\Box$} 

			\textcolor{blue}{\textbf{Megj:}} A $G$ komponense alatt sokszor nem csupán a $G$ csúcsainak egy $K$ részhalmazát, hanem a $K$ által feszített részgráfot értjük.

			\textcolor{orange}{\textbf{Megf:}} $G$ pontosan akkor összefüggő, ha egy komponense van. \raggedright \textcolor{blue}{$\Box$} 

			\textcolor{blue}{\textbf{Élhozzáadási lemma (ÉHL):}} Legyen $G$ irányítatlan gráf és $G' = G + e$. Ekkor az alábbi két esetből pontosan egy valósul meg.

			\textcolor{red}{\textbf{(1)}} $G$ és $G'$ komponensei megegyeznek, de $G'$-nek több köre van, mint $G$-nek.

			\textcolor{green}{\textbf{(2)}} $G$ és $G'$ körei megegyeznek, de $G'$-nek eggyel keveseb komponense van, mint $G$-nek.

		\subsection{Fák és erdők}

			\textcolor{blue}{\textbf{Def:}} A körmentes irányítatlan gráfot \textcolor{red}{erdőnek} nevezzük. Az öszefüggő, körmentes irányítatlan gráf neve \textcolor{red}{fa}.

			\textcolor{orange}{\textbf{Megf:}} $G$ erdő $\Longleftrightarrow  G$ minden komponense fa.

			\textcolor{purple}{\textbf{Példa:}}

			\textcolor{orange}{\textbf{Megf:}} (1) $P_n$ fa minden $n \geq 1$ egész esetén. (2) Fához egy új csúcsot egy éllel bekötve fát kapunk:

			\textcolor{orange}{\textbf{Lemma:}} $G$ $n$-csúcsú, $k$-komponensű erdő $\Rightarrow |E(G)| = n-k$.

			\textcolor{green}{\textbf{Biz:}} Építsük fel $G$-t a $\overline{K_n}$ üresgráfból az élek egyenkénti behúzásával. $G$ körmentes, ezért az ÉHL miatt minden lé zöld: behúzásakor 1-gyel csökken a kmponensek száma. A $\overline{K_n}$ üresgráfnak $n$ komponense van, $G$-nek pedig $k$. Ezért pontosan $n-k$ zöld élt kellett behúzni $G$ felépítéséhez. \raggedright \textcolor{blue}{$\Box$} 

			\textcolor{orange}{\textbf{Köv:}} Ha $F$ egy $n$-csúcsú fa, akkor élszáma $|E(F)|=n-1$.

			\textcolor{green}{\textbf{Biz:}} $F$ egy 1-komponensű erdő, így az előző Lemma alkalmazható $k=1$ helyettesítéssel.

			\textcolor{orange}{\textbf{Állítás:}} Tetsz. $n$-csúcsú $G$ gráf esetén az alábbi három tulajdonság közül bármely kettőből következik a harmadik. (a) $G$ körmentes. \qquad (b) $G$ összefüggő. \qquad (c) $|E(G)|=n-1$. 

			\textcolor{green}{\textbf{Biz:}} $(a)+(b) \Rightarrow (c): \checkmark$

			$(a)+(c)\Rightarrow (b)$: Építsük fel $G$-t élek egyenkénti behúzásával. $n-1$ él egyikánek behúzása se hoz létre kört, ezért az ÉHL miatt minden él zöld, és 1-gyel csökkenti a komponensszámot. Végül $n-(n-1)=1$ komponens marad, tehát $G$ összefüggő.

			$(b)+(c) \Rightarrow (a)$: Építsük fel $G$-t élek egyenkénti behúzásával. Mivel a komponensek száma végül 1 lesz, ezért $n-1$ zöld élt kellett behúzni. (c) miatt $G$ összes éle zöld, piros éle nincs. Az ÉHL miatt $G$ körmentes. \raggedright \textcolor{blue}{$\Box$} 

		\subsection{Fák további tulajdonságai}

			\textcolor{orange}{\textbf{Állítás:}} Legyen $F$ egy tetszőleges fa $n$ csúcson. Ekkor 

			(1) $(F-e)$-nek pontosan két komponense van $\forall e\in E(F)$-re.

			(2) $F$-nek pontosan egy $uv$-útja van $\forall u,v\in V(F)$-re.

			(3) $(F+e)$-nek pontosan egy köre van $\forall e\notin E(F)$-re.

			(4) Ha $n \geq 2$, akkor $F$-nek legalább két levele van.

			\textcolor{blue}{\textbf{Def:}} A $G$ irányítatlan gráf $v$ csúcsa \textcolor{red}{levél}, ha $d(v)=1$.

			\textcolor{green}{\textbf{Biz:}} (1): $F-e$ erdő, hisz körmentes. $F=(F-e)+e$, és mivel $F$ is körmentes, $e$ zöld az ÉHL miatt. Ezért $F$-nek 1-gyel kevesebb komponense van, mint $(F-e)$-nek. Mivel $F$-nek 1 komponense van, $(F-e)$-nek 2. \raggedright \textcolor{blue}{$\Box$} 
		
			\textcolor{green}{\textbf{Biz:}} (2): $F$ összefüggő, ezért van (legalább egy) $uv$-útja, mnodjuk $P$. Ezen $P$ út bármely $e$ élét elhagyva, a kapott $F-e$ grágnak (1) miatt két komponense van, melyek közül az egyik $u$-t, a másik $v$-t tartalmazza. Ezért ($F-e$)-ben nincs $uv$-út. Azt kaptuk, hogy $P$ minden éle benne van $F$ minden $uv$-útjában, ezért $F$-ben $P$-n kívül nincsmás $uv$-út. \raggedright \textcolor{blue}{$\Box$} 
		
			\textcolor{green}{\textbf{Biz:}} (3): Tfh $e=uv$. Minden $F$ körmentes, ezért $F+e$ minden köre $e$-ből és $F$ egy $uv$-útjából tevődik össze. Ezért $F+e$ köreinek száma megegyezik az $F$ fa $uv$-útjainka számával, ami (2) miatt pontosan 1. \raggedright \textcolor{blue}{$\Box$} 
		
			\textcolor{green}{\textbf{Biz:}} (4): (Algebrai út) A KFL miatt $\sum_{v\in V(G)}(d(v)-2)=\sum_{v\in V(G)}d(v)-2n=2(n-1)-2n=-2$. $F$ minden $v$ csúcsára $d(v) \geq 1$ teljesül, ezért $d(v) - 2 \geq -1$. A fenti összeg csak úgy lehet $-2$, ha $F$-nek legalább 2 levele van. \raggedright \textcolor{blue}{$\Box$} 
		
			\textcolor{green}{\textbf{Biz:}} (4): (Kombinatorikus út) Induljunk el $F$ egy tetszőleges $v$ csúcsából egy sétán, és haladjunk, amíg tununk. Ha sosem akadunk el, akkor előbb-utóbb ismétlődik egy csúcs, és kört találunk. Ezért elakadunk, és az csakis egy $v$-től különböző $u$ levélben történhet. Ha $d(v)=1$, akkor $v$ egy $u$-tól különböző levél. Ha $d(v) \geq 2$, akkor sétát indulhatjuk $v$-ből egy másik él mentén. Ekkor egy $u$-tól különböző levélben akadunk el. \raggedright \textcolor{blue}{$\Box$} 

		\subsection{Feszítőfák}
		
			Építsük fel a $G$ gráfot az élek egymás utái behúzásával, és az ÉHL szerinti kiszínezésével! Legyen \textcolor{green}{$G'$} a $G$ gráf piros élei törlésével keletkező feszítő részgráf! \textcolor{green}{$G'$} biztosan körmentes lesz, hiszen a zöld élek sosem alkottak kört a korábbi élekkel. \textcolor{green}{$G'$} minden \textcolor{green}{$K'$} komponense részhalmaza $G$ egy $K$ komponensének. Ha $K' \neq K$, akkor $G$-nek van olyan éle, ami kilép $K'$-ből. Ezen élek mind pirosak \textcolor{green}{$K'$} definíciója miatt. Legyen $e$ ezek közül az elsőnek kiszínezett. Az $e$ él nem tudott kört alkotni a korábbn kiszínezettekel, így nem leht piros: ellentmondás. Ezek szerint $G$ egy \textcolor{green}{$G'$} komponensei megegyeznek.

			\textcolor{orange}{\textbf{Köv:}} A $G$ gráf zöld élei olyan \textcolor{green}{$G'$} feszítő részgráfot alkotnak, ami erdő, és komponensei megegyeznek $G$ komponenseivel. \raggedright \textcolor{blue}{$\Box$} 

			\textcolor{blue}{\textbf{Def:}} $F$ a $G$ gráf \textcolor{red}{feszítőfája} (\textcolor{red}{ffája}), ha $F$ egy $G$-ből éltörlésekkel kapható fa.

			\textcolor{orange}{\textbf{Állítás:}} ($G$-nek van feszítőfája) $\Longleftrightarrow$ ($G$ összefüggő)

			\textcolor{green}{\textbf{Biz:}} $\Rightarrow$: Legyen $F$ a $G$ feszítőfája. $F$ összefüggő, és $V(F)=V(G)$, tehát $G$ bármely két csúcsa között vezet $F$-beli út.

			$\Leftarrow$: Építsük fel $G$-t az álek egyenkénti behúzásával és kiszínezésével. Láttuk, hogy a zöld élek egy \textcolor{green}{$F$} erdőt alkotnak, aminek egyetlen komponense van, hiszen $G$ is egykomponensű. Ezek szerint \textcolor{green}{$F$} olyan fa, ami $G$-ből éltörlésekkel kapható. \raggedright \textcolor{blue}{$\Box$} 
		
			\textcolor{blue}{\textbf{Megj:}} Ha egy nem feltétlenül összefüggő $G$ gráf éleit a fenti módon kiszínezzük, akkor a zöld élek $G$ minden komponensének egy \textcolor{green}{$F$} feszítőfáját alkotják. Nem összefüggő $G$ esetén a zöld élek alkotta feszítő részgráf neve a $G$ \textcolor{red}{feszítő erdeje}.

	\section{Minimális költségű feszítőfák}

		\subsection{Alapkörrendszer, alap vágás renszer}

			Adott egy $G$ gráf és $G$-nek egy $F$rögzített feszítőfája. Ekkor $G$ minden éléhez $F$ meghatározza $G$ éleinek egy fontos részhalmazát. Attól függően, hogy az adott él $F$-hez tartozik-e vagy sem, különböző fajta részhalmazról van szó.

			\textcolor{blue}{\textbf{Def:}} A $G$ gráf $F$ feszítőfájának $f$ éléhez tartozó \textcolor{red}{alap vágást} $G$ azon élei alkotják, amik az $F-f$ két komponense között futnak. Az $e \in E(G) \backslash E(F)$ éléhez tarozó \textcolor{red}{alapkör} pedig az $F+e$ köre.

			\textcolor{orange}{\textbf{Megf:}} Tfh $f \in F$ és $e \in E(G) \backslash E(F)$. Ekkor ($F-f+e$ ffa) $\Longleftrightarrow$ ($f$ benne van $e$ alapkörében) $\Longleftrightarrow$ ($e$ benne van $f$ alap vágásában).

			\textcolor{orange}{\textbf{Köv:}} Az $e \in E(G) \backslash E(F)$ alapkörét $e$ mellett azon $F$-beli élek alkotják, amelyek alapvágása $e$-t tartalmazza. Az $f \in F$ alapvágást $f$ mellett a $G$ azon élei alkotják, amelyek alapköre $f$-t tartalmazza.

		\subsection{Minimális költségű feszítőfa}

			\textcolor{blue}{\textbf{Def:}} Adott a $G = (V,E)$ irányítatlan gráf élein a $k:E \rightarrow \mathbb{R}_+$ költségfüggvény. Az $F \subseteq E$ élhalmaz \textcolor{red}{költsége} az $F$-beli élek összköltsége: $k(F) = \sum_{f\in F}k(F)$.

			Az $F \subseteq E$ élhalmaz $G$-ben \textcolor{red}{minimális költségű feszítőfa} (\textcolor{red}{mkffa}), ha 

			(1) $(V,F)$ a $G$ feszítőfája, és
			
			(2) $k(F) \leq k(F')$ teljesül a $G$ bármely $(V,F')$ feszítőfájára.
			
			Az $F \subseteq E$ élhalmaz $G$-ben \textcolor{red}{minimális költségű feszítő erdeje}, ha 

			(1) $(V,F)$ a $G$ feszítő erdeje, és

			(2) $k(F) \leq k(F')$ teljesül a $G$ bármely $(V,F')$ feszítő erdejére.

			\textcolor{red}{\textbf{Cél:}} Hatékony eljárás mkffa keresésére.
			
			\textcolor{purple}{\textbf{Ötlet:}} Keressük a feszítőfát a tanult módon, az élk egyenkénti behúzásával, az ÉHL szerint zöldre színezett élek halmazaként. 

			Zöld él: olyan él, ami nem alkot kört a korábban felépített élekkel.

			\textcolor{green}{\textbf{Mohó stratégia:}} A feszítőfa építésekor költség szerint növekvő sorrendben döntsünk az élekről, hátha mkffát kapunk a végén.

			\textcolor{blue}{\textbf{Kruskal-algoritmus:}} \underline{Input}: $G=(V,E)$ és $k:E \rightarrow \mathbb{R}_+$ költségfüggvény. \underline{Output}: $F \subseteq E$ \underline{Működés}: Tfh $k(e_1) \leq k(e_2) \leq \dots \leq k(e_m)$, ahol $E=\{e_1,e_2,\dots, e_m\}$. Legyen $F_0=0$, és $i=1,2,\dots,m$-re

				\begin{equation*}
					F_i :=\begin{cases}
						F_{i-1}\cup \{e_i\} & \text{ha $F_{i-1}\cup \{e_i\}$ körmentes.} \\
						F_{i-1} & \text{ha $F_{i-1}\cup \{e_i\}$ tartalmaz kört. \;\;\; $F:=F_m$}
					\end{cases}
				\end{equation*}

		\subsection{Minimális költségű feszítőfák struktúrája}

			$G=(V,E)$ gráf és $k:E \rightarrow \mathbb{R}_+$ költségfüggvény esetén legyen $G_c$ a legfeljebb $c$ költségű élek alkotta feszítő részgráfja $G$-nak: $G_c = (V,E_c)$, ahol $E_c := \{e \in E : k(e)\leq c\}$.

			\textcolor{orange}{\textbf{Megf:}} A $G$ gráfon futtotott Kruskal-algoritmus outputja tartalmazza $G_c$ egy feszítő erdejét minden $c \geq 0$ esetén.

			\textcolor{green}{\textbf{Biz:}} A Kruskal-algoritmus a legfeljebb $c$ költségű ($E_c$-beli) éleket hamarabb dolgozza fel, mint a $c$-nél drágábbakat. Ezért $E_c$ összes élének feldolgozása után pontosan azt az állapotot érjük el, mintha a Kruskal-algoritmust a $G_c$ frágon futtattunk volna. Korábban (az ÉHL előtt) láttuk, hogy az utóbbi algoritmus outputja $G_c$ egye feszítő erdeje. \raggedright \textcolor{blue}{$\Box$} 

			\textcolor{orange}{\textbf{Lemma:}} Tfh $F=\{f_1,f_2,\dots,f_l\}, k(f_1) \leq k(f_2) \leq \dots \leq k(f_l)$ és $F\cap E_c$ a $G_c$ egy feszítő erdeje $\forall c \geq 0$-ra. Tfh $F' = \{f'_1, f'_2,\dots,f'_l\}$ a $G$ egy feszítő erdejének élei, és $k(f'_1) \leq k(f'_2) \leq \dots \leq k(f'_l)$. Ekkor $k(f_i) \leq k(f'_i)$ teljesül $\forall 1 \leq i \leq l$ esetén, így $k(F) \leq k(F')$.
			
			\textcolor{green}{\textbf{Biz:}} Indirekt: tfh $k(f_i) > k(f'_i) = c$. Ekkor $|E_c \cap F| <i$, így a feltevés miatt $E_c \cap F$a $G_c$ egy $i$-nél kevesebb élű feszítő erdeje. Az $f'_1, f'_2, \dots, f'_i$ élek is mind $E_c$-beliek, és többen vannak az $E_c\cap F$ feszítő erdő élszámánál. Tehát $f'_1, f'_2, \dots, f'_i$ nem lehet körmentes, így $f'_1, f'_2, \dots, f'_l$ sem. Ez ellentmondás. Tehát $k(f_i) \leq k(f'_i) \:\forall i$. Ezért $k(F) = \sum_{i=1}^{l}k(f_i) \leq \sum_{i=1}^{l}k(f'_i)=k(F')$. \raggedright \textcolor{blue}{$\Box$} 

			\textcolor{orange}{\textbf{Köv:}} (1) A Kruskal-algoritmus outputja a $G$ gráf egy minimális költségű feszítő erdeje. 
			
			\textcolor{green}{\textbf{Biz:}} Legyen $F$ a Kruskal-algoritmus outputja. A megfigyelés miatt $F \cap E_c$ a $G_c$ feszítő erdeje $\forall c \geq 0$-ra, ezért a Lemma szerint $k(F) \leq k(F')$ teljesül $G$ tetszőleges $F'$ feszítő erdejére. \raggedright \textcolor{blue}{$\Box$} 

			\textcolor{orange}{\textbf{Köv:}} (2) Az $F'$ élhalmaz pontosan akkor minimális költségű feszítő erdeje $G$-nek, ha $F' \cap E_c$ a $G_c$ egy feszítő erdeje minden $c \leq 0$-ra.

			\textcolor{green}{\textbf{Biz:}} A Lemma bizonyítja az elégfégességet.

			\textcolor{green}{\textbf{Biz:}} A szükségességhez tfh $F' \cap E_c$ nem feszítő erdeje $G_c$-nek, és legyen $F$ a Kruskal-algoritmus outputja. Mivel $F \cap E_c$ a $G_c$ feszítő erdeje, ezért $|F \cap E_c| > |F' \cap E_c|$, így $k(f_i) < k(f'_i)$ teljesül legalább egy $i$-re, és minden $j$-re $k(k_j) \leq k(f'_i)$. Innen $k(F) < k(F')$. \raggedright \textcolor{blue}{$\Box$}

			\textcolor{orange}{\textbf{Köv:}} (3) Ha a $G$ gárf összefüggő, akkor $G$ feszítő erdeje a $G$ feszítő fája, így a Kruskal-algoritmus mkffát talál. A (2) következmény pedig $G$ mkffáit karakterizálja.
		
		\subsection{Az ötödik elem}
		
			Algoritmusok megadásakor öt dologra figyelünk:

			Input \checkmark, Output \checkmark, Működés \checkmark, Helyesség \checkmark, Lépésszám. 

			Utóbbiról nem volt szó a Kruskal-algoritmus esetében. 

			Tfh $n$ ill. $m$ a $G$ csúcsai ill. élei száma. 

			A Kruskal-algoritmus két részből áll: 

			\;\;\;\;\;\textcolor{blue}{1.} Élek költség szerinti sorbarendezése 

			\;\;\;\;\;\textcolor{blue}{2.} Döntés az egyes élekről a fenti sorrendben. 
			
			\textcolor{blue}{1.} $m$ szám sorbarendezéséhez a buborékrendezés legfeljebb $\binom{m}{2}$ összehasonlítást használ.
			
			\textcolor{blue}{1.} $n$ csúcsú $G$ gráf esetén egy élről döntés megoldható $konst \cdot \log_2n$ lépésben az adatstruktúra karbantartásával együtt is. Az összes döntéshez tehát elegendő $konst \cdot n \cdot \log_2n$ lépés. A Kruskal-algoritmus lépésszáma ezért felülről becsülhető $konst \cdot (n+m) \cdot \log_2(n+m)$-mel.
		
	\section{Gráfbejárások és legrövidebb utak}
		
		\subsection{Általános gráfbejárás $\&$ BFS}

			A gráfbejárási algoritmus az inputgráf csúcsait és éleit fedezi fel. Minden csúcs az eléretlen $\rightarrow$ elért $\rightarrow$ befejezett állapotokat veszi fel. A bejárás akkor ér véget, amint minden csúcs befejezetté vált. 

			1. Van elért csúcs. Választunk egyet, mondjuk $u$-t.
				
			(1a) Ha van olyan $uv$ él, amire $v$ eléretlen, akkor $v$ elérté válik.

			(1b) Ha nincs ilyen $uv$ él, akkor $u$ befejezetté válik.

			2. Nincs elért csúcs.

			(2a) Ha van eléretlen $u$ csúcs, akkor $u$-t elértté tesszük.

			(2b) Ha nincs eléretlen csúcs (azaz $\forall$ csúcs fejezett), akkor END.

			\textbf{\textcolor{blue}{Szélességi bejárás (BFS) szabálya:}}

			Az 1. esetben mindeg a legkorábban elért $u$-t választjuk.

			\textbf{Input:} $G = (V,E)$ (ir/ir.tatlan) gráf, ($v \in V$ gyökérpont\footnote{A gyökérben kezdetben elért állapotú, ezért kivétel az általános szabály alól.}).

			\textbf{Output:} (1) A csúcsok elérési és befejezési sorrendje. (2) Az élek osztályozása:

			\textbf{\textcolor{green}{faél:}} Olyan él, ami mentén egy csúcs elértté vált.

			$uv$ \textbf{\textcolor{brown}{előreél:}} nem faél, de $u$-ból $v$-be faélekből irányított út vezet.

			$uv$ \textbf{\textcolor{blue}{visszaél:}} $v$-ből $u$-ba faélekből irányított út vezet.

			\textbf{\textcolor{red}{keresztél:}} minden más él ($u$ és $v$ közt nincs leszármazott viszony).

			(3) A \textbf{\textcolor{green}{bejárás fája:}} a faélek alkotta részgráf. (A bejárás fája valójában egy gyökereiből kifelé irányított erdő.)

			\textbf{\textcolor{orange}{Megf:}} Irányítatlan esetben az előreél és a visszaél ugyanazt jelenti.

			\textbf{\textcolor{blue}{Terminológia:}} Ha a bejárás fájában $u$-ból $v$-be irányított út vezet, akkor $u$ a $v$ őse és $v$ az $u$ leszármazottja. A faél és az előreél tehát őszből leszármazottba, a visszaél leszármazottból ősbe vezet.

		\subsection{A BFS tulajdonságai}
			
			Nézzük meg egy \textbf{irányított} gráf BFS bejárását is.

			\textbf{\textcolor{Orange}{Állítás:}} Tfh $G=(V,E)$ BFS bejárása után a csúcsok elérési sorrendje $v_1,v_2,\dots,v_n$. Ekkor az alábbiak teljesülnek.

			(1) Ha $i < j$, akkor $v_i$-t hamarabb fejezük be, mint $v_j$-t, továbbá $v_i$ gyerekei megleőzik $v_j$ gyerekeit az elérési sorrendben.

			\textbf{\textcolor{green}{Biz:}} A $v_i$-t befejezésének pillanatában $v_i$ minden gyereke elért, de $v_j$-nek még egy gyereke sem az. Ezért $v_j$ gyerekeit a $v_i$ csúcs befejezése után érjük el, majd ezt követően fejezzük be $v_j$-t. \raggedright \textcolor{blue}{$\Box$} 

			(2) \textbf{Az elérési és befejezési sorrend (BFS esetén) megegyezik.}

			\textbf{\textcolor{green}{Biz:}} Ha $v_i$-t korábban érjük el, mint $v_j$-t, akkor (1) miatt $v_i$-t korábban is fejezzük be $v_j$-nél. Ezért bármely két csúcs sorrendje ugyanaz az elérési sorrendben mint befejezési sorrendben. Tehát az elérési sorrendnek meg kell egyeznie a befejezési sorrenddel. \raggedright \textcolor{blue}{$\Box$} 

			(3) \textbf{Gréfél nem ugorhat át falét:} ha $k < i < j \leq l$ és  $v_i v_j$ faél, akkor $v_k v_l$ nem lehet gráfél.

			\textbf{\textcolor{green}{Biz:}} Ha $v_k v_l \in E(G)$, akkor $v_l$ szülője $v_k$ vagy egy $v_k$-t megelőző csúcs. (1) miatt $v_j$ szülője sem következhet $v_k$ után, vagyis $v_i$ nem lehet $v_j$ szülője.

			(4) \textbf{Nincs előreél.} (Irányítatlan eset: csak faél és keresztél van.) 

			\textbf{\textcolor{green}{Biz:}} Indirekt: ha $v_i v_j$ előreél lenne, akkor $v_i$-ből $v_j$-be irányított út vezetne a BFS-fában, és $v_i v_j$ ennek a faélekből álló útnak az utolsó élét átugraná. \raggedright \textcolor{blue}{$\Box$} 

			(5) Ha a BFS-fában $k$-élű irányított út vezet $u$-ból $v$-be, akkor $G$-ben nincs $k$-nál kevesebb élű $uv$-út.

			\textbf{\textcolor{green}{Biz:}} Ha lenne a BFS fa-beli útnál kevesebb elű út $G$-ben, akkor lenne olyan gráfél, ami faélt ugrik át. \raggedright \textcolor{blue}{$\Box$} 

			(6) \textbf{A BFS-fa egy legrövidebb utak fája:} a BFS-fa $v_1$ gyökeréből bármely $v_i$ csúcsba vezető faút a $G$ egy legkevesebb élű $v_1 v_i$-útja.

		\subsection{Legrövidebb utak}
				
			\textbf{\textcolor{blue}{Def:}} Adott $G$ (ir) gráf és $l : E(G) \rightarrow \mathbb{R}$ hosszfüggvény esetén egy \textcolor{red}{$P$ út hossza} a $P$ éleinek összhossza: $l(P) = \sum_{e\in E(P)} l(e)$.

			Az $u$ és $v$ csúcsok \textcolor{red}{távolsága} a legrövidebb $uv$-út hossza: $dist_l(u,v):=$ min$\{l(P):P \;uv$-út$\}$ $(\nexists uv$-út$\Rightarrow dist_l (u,v)= \infty.)$ Az $l$  hosszfüggvénye \textcolor{red}{nemnegatív}, ha $l(e) \geq 0$ teljesül minden $e$ élre. Az $l$ hosszvüggvény \textcolor{red}{konzervatív}, ha $G$-ben $\nexists$ negatív összhosszú ir. kör.

			\textbf{\textcolor{red}{Cél:}} Legrövidebb út keresése irányított/irányítatlan gráfban.

			\textbf{\textcolor{orange}{Megf:}} Ha $l(e) = 1 $ a $G$ minden $e$ élére, akkor $l(P)$ a $P$ élszáma. Ezért a BFS-fa minden gyökérből elérhető csúcsba tartalmaz egy legrövidebb utat a gyökérből elérhető csúcsba tartalmaz egy legrövidebb utat a gyökérből, azaz a szélességi bejárás tekinthető egy legrövidebb utat kereső algoritmusnak is. 

			\textbf{\textcolor{blue}{Def:}} Adott $G$ (ir) gráf, $l : E(G) \rightarrow \mathbb{R}$ hosszfüggvény és $r \in V(G)$. \textcolor{red}{$(r,l)$-felső becslés} olyan $f: V(G) \rightarrow \mathbb{R}$ függvény, ami felülről becsli minden csúcs $r$-től mért távolságát: $dist_l (r,v) \geq f(v) \forall v \in V(G)$.

			\textcolor{red}{Triviális} $(r,l)$-felső becslés:
			$
				f(v) = \begin{cases}
					0 & v = r \\
					\infty & v \neq r
				\end{cases}
			$

			\textcolor{red}{Pontos} $(r,l)$-felső becslés: $f(v) = dist_l(r,l)\; \forall v \in V(G)$.

		\subsection{Az elméleti javítás}

			\textbf{\textcolor{blue}{Def:}} Tfh $f$ egy $(r,l)$-felső becslés és $uv \in E(G)$. Az $f$ \textcolor{red}{$uv$-elméleti javítása} az az $f'$, amire 
			$
				f'(z) = \begin{cases}
					f(z) & z \neq v \\
					min\{f(v), f(u) + l(uv)\} & z = v
				\end{cases}
			$

			\textbf{\textcolor{orange}{Megf:}} Tfh az $l:E(G)\rightarrow \mathbb{R}$ hosszfüggvény konzervatív és $f(r) = 0$. 
			
			Ekkor (1) Az $f (r,l)$-felső becslés élmenti javítása mindig $(r,l)$-felső becslést ad.

			\textbf{\textcolor{green}{Biz:}} Azt kell megmutatni, hogy van  olyan $rv$-út, aminek a hossza legfeljebb $f(u) + l(uv)$. Ha egy legrövidebb $ru$-utat kiegészítünk az $uv$ éllel, akkor olyan $rv$-élsorozatot kapunk, aminek az összhossza $dist_l (r,u) + l(uv) \leq f(u) + l(uv)$. ,,Könnyen" látható, hogy  az élhosszfüggvény konzervativitása miatt ha van $x$ összhosszúságú $rv$-élsorozat, akkor van legfeljebb $x$ összhosszúságú $rv$-út is. Ezek szerint van legfeljebb $f(u) + l(u,v)$ hosszúságú $uv$-út is, azaz az érdemi élmenti javítás után szintén $(r,l)$-felső becslést kapunk.  \raggedright \textcolor{blue}{$\Box$} 

			(2) $f (r,l)$-felső becslés (pontosan) $\Longleftrightarrow$ ($f$-en $\nexists$ érdemi élmenti javítás).

			\textcolor{green}{\textbf{Biz:}}  $\Rightarrow$: Ha $f$ pontos, akkor biztosan nincs rajta érdemi élmenti javítás: ha volna, akkor egy felső becslés a pontos érték alá csökkenne, így az élmenti javítás nem  $(r,l)$-felső becslést eredményezne. $\Leftarrow$: Legyen $v \in V(G)$ tetsz, és legyen $P$ egy legrövidebbb $rv$-út. A $P$ egyik éle mentén sincs érdemi élmenti javítás, ezért $P$ minden $u$ csúcsára pontos a felső becslés: $f(u) = dist_l(r,u)$. Ez igaz az út utolsó csúcsára, a tetszőlegesen választott $v$-re is. \raggedright \textcolor{blue}{$\Box$}

			\textbf{\textcolor{orange}{Köv:}} Adott $G$, konzervatív $l$ és $r \in V(G)$ esetén ha kiindulunk a triviális $(r,l)$-felső becslésből, és addig végzünk émj-kat, amíg lehet, akkor a végén megkapjuk minden csúcs $r$-től való távolságát.

			\textbf{\textcolor{red}{Itt a jegyzet 17. oldaláról az utolsó kettő pont hiányzik, mivel nem tudom, hogy mennyire lényegesek.}}

			\textbf{\textcolor{blue}{Def:}} Tfh $f$ egy $(r,l)$-felső becslés és $uv \in E(G)$. Az $f$ \textcolor{red}{$uv$-élmenti javítása} az az $f'$, amire 
			$
				f'(z) = \begin{cases}
					f(z) & z \neq v \\
					min\{f(v), f(u) + l(uv)\} & z = v
				\end{cases}
			$

			\textbf{\textcolor{orange}{Megf:}} Tfh az $l : E(G) \rightarrow \mathbb{R}$ hosszfüggvény konzervatív és $f(r) = 0$. Ekkor (1) Az $f (f,l)$-felső becslés élmenti javítása mindig $(r,l)$-felső becslést ad. (2) $f(r,l)$-felső becslés (pontosan) $\Leftrightarrow$ ($f$-en $\nexists$ érdemi élmenti javítás).

			\textbf{\textcolor{blue}{Dijkstra-algoritmus:}} \underline{Input:} $G = (V,E), l : E \rightarrow \mathbb{R}_+, r \in V$. \underline{Output:} $dist_l(r,v) \forall v \in V$ \underline{Működés:} $U_0 := \emptyset, f_0$ a triviális. $(r,l)$-felső becslés. 
			
			Az $i$-dik fázis:

			1. Legyen $U_i := U_{i-1} \cup \{u_i\}$, ahol $u_i$ olyan csúcs a $V  \setminus  U_{i-1}$ halmazból, amelyre $f_{i-1}(v)$ minimális.

			2. $f_i:f_{i-1}$ élmenti javítása minden $U_i$-ből kivezető $u_ix$ élen. Output: $f_{|V|}$. Megjelöljük a végső $f_{|V|}(V)$ értékeket beállító éleket.
			
		\subsection{Dijkstra, egy példán}

			\textbf{\textcolor{blue}{Dijkstra-algoritmus:}} \underline{Input:} $G = (V,E), l : E \rightarrow \mathbb{R}_+, r \in V$. \underline{Output:} $dist_l(r,v) \forall v \in V$ \underline{Működés:} $U_0 := \emptyset, f_0$ a triviális. $(r,l)$-felső becslés. 
				
			Az $i$-dik fázis:

			1. Legyen $U_i := U_{i-1} \cup \{u_i\}$, ahol $u_i$ olyan csúcs a $V  \setminus  U_{i-1}$ halmazból, amelyre $f_{i-1}(v)$ minimális.

			2. $f_i:f_{i-1}$ élmenti javítása minden $U_i$-ből kivezető $u_ix$ élen. Output: $f_{|V|}$. Megjelöljük a végső $f_{|V|}(V)$ értékeket beállító éleket.
			
			\textcolor{orange}{\textbf{Megf:}} Ha a $v$-be vezet megjelölt él, akkor vezet $r$-ből $v$-be megjelölt éleken út, és ennek hozza megegyezik $f_{|V|} (v)$-vel.

			\textcolor{green}{\textbf{Biz:}} $f_{|V|} (r) = 0$, és a megjelölt élek mentén haladva az $f_{|V|}$ érték az élhosszal növekszik. \raggedright \textcolor{blue}{$\Box$}

			\textcolor{orange}{\textbf{Köv:}} Ha a Dijsktra-algoritmus helyes, akkor az algoritmus végén a megjelölt élek egy legrövidebb utak fáját alkotják $r$ gyökérrel.

		\subsection{Dijkstra helyessége}
				
			\textcolor{orange}{\textbf{Megf:}} Tfh $u_1, u_2, \dots, u_n$ a $G$ csúcsainak sorrendje a Dijkstra-algoritmus végrehajtása után. 

			(1) Ekkor $f_{|V|}(u_i) \leq f_{|V|}(u_{i+1})$ teljesül $\forall 1 \leq i \leq n$.

			\textcolor{green}{\textbf{Biz:}} Az $i$-dik fázisban $f_i(u_i) \leq f_i(u_{i+1})$ teljesült az $u_i$ választása miatt. Ezek után $f_i(u_i)$ már nem változott: $f_{|V|}(u_i) = f_i(u_i)$. Ugyan $f_i(u_{i+1})$ még csökkenhetett, de csak az $u_iu_{i+1}$ él mentén történt javítás miatt, hiszen az $(i+1)$-dik fázisban $u_{i+1}$ bekerült az $U_i$ halmazba, és a hozzá tartozó $(r,l)$-fb már nem csökken tovább. Ekkor $f_{i+1}(u_{i+1}) = min \{f_i(u_{i+1}),f_i(u_i)+l(u_iu_{i+1})\} \geq f_i(u_i)$, mivel $l(u_iu_{i+1}) > 0$. Ezért $f_{|V|}(u_i) = f_i(u_i) \leq f_{i+1}(u_{i+1}) = f_{|V|}(u_{i+1})$ \raggedright \textcolor{blue}{$\Box$}

			(2) $f_{|V|}(u_1) \leq f_{|V|}(u_2) \leq \dots \leq f_{|V|}(u_n)$

			(3) A Dijsktra-algoritmus outputjaként kaptt $f_{|V|}$-n élmenti javítás nem tud változtatni.
 
			\textcolor{green}{\textbf{Biz:}} Tegyük fel, hogy $u_iu_j \in E(G)$ a $G$ egy tetszőleges éle. Ha $i > j$, akkor (2) miatt $f_{|V|}(u_i) \geq f_{|V|}(u_j)$, ezért az $u_iu_j$ mentén történő javítás nem tudja $f_{|V|}(u_j)$-t csökkenteni, hisz $l(u_iu_j)$ pozitív. Ha pedig $i < j$, akkor az $i$-dik fázisban megrörtént az $u_iu_j$ mentén történő javítás, és ezt követően $f(u_i)$  nem váltorott, azaz $f_{|V|}(u_i) = f_i(u_i)$. A másik $(r,l)$-felső becslés pedig csak tovább csökkenhetett a késpbbi émj-ok során $f_{|V|}(u_j) \leq f_i(u_j)$. Ezért az $u_iu_j$ él mentén sem az $i$-dik fázisban, sem később nincs érdemi javítás. \raggedright \textcolor{blue}{$\Box$}

			\textcolor{orange}{\textbf{Tétel:}} A Dijsktra-algoritmus helyesen működik, azaz $G$ minden csúcsára igaz, hogy $dist(r,v) = f_{|V|}(v)$.

			\textcolor{green}{\textbf{Biz:}} A Dijsktra-algoritmus az $f_0$ triviális $(r,l)$-felső becslésből indul ki, és élmenti javításokat alkalmaz. Így minden $f_i$ (speciálisan $f_{|V|}$ is) $(r,l)$-felső becslés lesz. A fenti (3)-as megfigyelés miatt $f_{|V|}$-n nem  végezhető érdemi élmenti javítás. Ezért egy korábbi (2)-es megfigyelés miatt $f_{|V|}$ pontos $(r,l)$-felső becslés, azaz $f_{|V|}(v) = dist_l(r,v) \forall v \in V(G)$. \raggedright \textcolor{blue}{$\Box$}

			\textbf{\textcolor{blue}{,,Lépésszámanalízis":}} Ha a $G$ gráfnak $n$ csúcsa és $m$ éle van, akkor a Dijkstra-algoritmus $n$-szer keresi meg legfeljebb $n$ szám minimumát, ami összességében legfeljebb $konst \cdot n^2$ lépést igényel. Ezen kívül legfeljebb $m$ élmenti javítást véges, ami $konst' \cdot m$ lépés. Összességében tehát legfeljebb $konst'' \cdot (n^2 + m)$ lépésre van szükség, az algoritmus hatékony. 

	\section{Legrövidebb utak, DFS, PERT}
		
	\section{Euler-séták és Hamilton-körök}
	
	\section{Síkgráfok}
	
	\section{Lineáris egyenletrendszerek}
	
	\section{Az $\mathbb{R}^n$ tér alaptulajdonságai}
	
	\section{Altér bázisa és dimenziója}
	
	\section{Négyzetes mátrix determinánsa}
	
	\section{Mátrixműveletek és lineáris leképezések}
	
	\section{Mátrix rangja és inverze}
	
	\section{Mátrixegyenletek}
	
\end{document}
