\documentclass[12pt]{article}
\usepackage{amssymb}
\usepackage{amsmath}
\usepackage{xcolor}
\usepackage{geometry}
	\geometry{a4paper, total={170mm,257mm}, left=20mm, top=20mm, }
	
\renewcommand*\contentsname{Tartalomjegyzék}

\AddToHook{cmd/section/before}{\clearpage}


\begin{document}
\begin{titlepage}
	\centering \vfill
	{\textsc{Budapesti Műszaki és Gazdaságtudományi Egyetem} \par} \vspace{7cm}
	{\huge\bfseries A számítástudomány alapjai\par} \vspace{0.5cm}
	{\large \textsc{Összefoglaló jegyzet}\par} \vspace{1.5cm}
	{\Large\itshape Készítette: Illyés Dávid\par} \vfill

	\noindent\fbox{%
    	\parbox{140mm}{
			Ez  a jegyzet nagyon hasonlóan van struktúrálva az előadás jegyzetekhez és fő célja, hogy olyan módon adja át a "A Programozás Alapjai 1" nevű tárgy anyagát, hogy az teljesen kezdők számára is könnyen megérthető és megtanulható legyen. 
   		}
	}
	
	\vfill {\large \today\par}
\end{titlepage} 
\tableofcontents
\addtocontents{toc}{~\hfill\textbf{Oldal}\par}

	\section{A gráfelmélet alapjai}
		\subsection{Mi a gráf?}
			
			\textbf{Def:} $G = (V,E)$ \textcolor{red}{egyszerű, irányítatlan gráf}
			
			\textbf{Példa:} ha $ V \neq 0$ és $E \subseteq \binom{V}{2}$, ahol $\binom{V}{2} = \{\{u,v\} : u,v \in V, u \neq v\}$.
			
			$V$ a $G$ \textcolor{red}{csúcsainak} (vagy \textcolor{red}{(szög)pontjainak}), $E$ pedig G \textcolor{red}{éleinek} halmaza. 
			
			\textbf{Példa:} $G = (\{a,b,c,d\},\{a,b\},\{a,c\},\{b,c\},\{b,d\})$
			
			\textbf{Def:} A $G = (V,E)$ gráf \textcolor{red}{diagramja} a $G$ egy olyan lerajzolása, amiben $V$-nek a sík különböző pontjai felelnek meg, és $G$ minden $\{u,v\}$ élének egy $u$-t és $v$-t összekötő görbe felel meg.
			
			\textbf{Terminológia \& konvenciók:} Gráf alatt rendszerint egyszerű, irányítatlan gráfot értünk. Ha $G$ egy gráf, akkor $V(G)$ a $G$ csúcshalmazát, $E(G)$ pedig $G$ élhalmazát jelöli, azaz $G = (V(G),E(G))$. Az $e = \{u,v\}$ élt röviden $uv$-vel jelöljük. 
			
			Ekkor $e$ az $u$ és $v$ csúcsokat \textcolor{red}{köti össze}. Továbbá $u$ és $v$ az $e$ \textcolor{red}{végpontjai}, amelyek az $e$ élre \textcolor{red}{illeszkednek}, és $e$ mentén \textcolor{red}{szomszédosak}.
		
		\subsection{Multigráfok és irányított gráfok}

			\textbf{Megj:} Ha egy gráf nem egyszerű, akkor lehetnek \textcolor{red}{párhuzamos élei, hurokélei} vagy akár párhuzamos hurokélei is.
			
			\textbf{Def:} Az \textcolor{red}{irányított gráf} olyan gráf, aminek minden éle irányított.



		\subsection{Handshaking lemma}
		\subsection{Komplementer és izomorfia}
		\subsection{Gráfoperációk}
		\subsection{Háromféle elérhetőség, összefüggőség}
		\subsection{Gráfok összefüggősége a gyakorlatban}
		\subsection{Fák és erdők}
		\subsection{Fák további tulajdonságai}
		\subsection{Feszítőfák}
		
		
	\section{Minimális költségű feszítőfák}
		\subsection{Alapkörrendszer, alap vágás renszer}
		\subsection{Minimális költségű feszítőfa}
		\subsection{Minimális költségű feszítőfák struktúrája}
		\subsection{Az ötödik elem}
		\subsection{Mkkffák egy villamosmérnöki alkalmazása}
		
	\section{Gráfbejárások és legrövidebb utak}
		
	\section{Legrövidebb utak, DFS, PERT}
		
	\section{Euler-séták és Hamilton-körök}
	
	\section{Síkgráfok}
	
	\section{Lineáris egyenletrendszerek}
	
	\section{Az \(\mathbb{R}^n\) tér alaptulajdonságai}
	
	\section{Altér bázisa és dimenziója}
	
	\section{Négyzetes mátrix determinánsa}
	
	\section{Mátrixműveletek és lineáris leképezések}
	
	\section{Mátrix rangja és inverze}
	
	\section{Mátrixegyenletek}
	
\end{document}
