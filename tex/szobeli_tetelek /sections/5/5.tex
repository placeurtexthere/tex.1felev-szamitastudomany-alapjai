\documentclass[../../szobeli.tex]{subfiles}

\begin{document}

\begin{center}
    \noindent\fbox{%
    	\parbox{160mm}{
			Gráfút hossza, gráfcsúcsok távolsága, nemnegatív és konzervatív hosszfüggvény, triviális és pontos $(r,l)$\textbf{-felső becslés, \underline{élmenti javítás.} Dijkstra-algoritmus működése}, Ford-algoritmus \underline{helyessége} és lépésszáma. Legrövidebb utak fájának létezése.
        }
    }
\end{center}

    \begin{itemize}
        \item \textbf{\textcolor{blue}{Def:}} Adott $G$ (ir) gráf és $l : E(G) \rightarrow \mathbb{R}$ hosszfüggvény esetén egy \textcolor{red}{$P$ út hossza} a $P$ éleinek összhossza: $l(P) = \sum_{e\in E(P)} l(e)$.

        Az $u$ és $v$ csúcsok \textcolor{red}{távolsága} a legrövidebb $uv$-út hossza: $dist_l(u,v):=$ min$\{l(P):P \;uv$-út$\}$ $(\nexists uv$-út$\Rightarrow dist_l (u,v)= \infty.)$ Az $l$  hosszfüggvénye \textcolor{red}{nemnegatív}, ha $l(e) \geq 0$ teljesül minden $e$ élre. Az $l$ hosszvüggvény \textcolor{red}{konzervatív}, ha $G$-ben $\nexists$ negatív összhosszú ir. kör.

        \textbf{\textcolor{red}{Cél:}} Legrövidebb út keresése irányított/irányítatlan gráfban.

        \textbf{\textcolor{orange}{Megf:}} Ha $l(e) = 1 $ a $G$ minden $e$ élére, akkor $l(P)$ a $P$ élszáma. Ezért a BFS-fa minden gyökérből elérhető csúcsba tartalmaz egy legrövidebb utat a gyökérből elérhető csúcsba tartalmaz egy legrövidebb utat a gyökérből, azaz a szélességi bejárás tekinthető egy legrövidebb utat kereső algoritmusnak is. 

        \textbf{\textcolor{blue}{Def:}} Adott $G$ (ir) gráf, $l : E(G) \rightarrow \mathbb{R}$ hosszfüggvény és $r \in V(G)$. \textcolor{red}{$(r,l)$-felső becslés} olyan $f: V(G) \rightarrow \mathbb{R}$ függvény, ami felülről becsli minden csúcs $r$-től mért távolságát: $dist_l (r,v) \geq f(v) \forall v \in V(G)$.

        \textcolor{red}{Triviális} $(r,l)$-felső becslés:
        $
            f(v) = \begin{cases}
                0 & v = r \\
                \infty & v \neq r
            \end{cases}
        $

        \textcolor{red}{Pontos} $(r,l)$-felső becslés: $f(v) = dist_l(r,l)\; \forall v \in V(G)$.\
        
        \item Adott $G=(V,E)$ irányított gráf és egy $l:E \rightarrow \mathbb{R}$ élhosszfv. 
        Egy $G$-beli irányított út hossza az út éleinek összhossza, $dist_l(n,v)$ pedig az irányított $uv$-utak közül a legrövidebb hosszát jelöli. 

        Az $l$ hosszfv konzervatív ha nincs $G$-ben negatív összhosszúságú irányított kör. 
        
        Adott $G=(V,E)$ irányított gráf $r\in v$ és egy $l:E\rightarrow R$ élhosszfv. Az $f:v \rightarrow R$ függvényt $(r,l)$-felső becslésnek nevezzük, ha $f(r)=0$ és $f(v) \leq dist_l(r,v)$ teljesül $G$ minden $v$ csúcsára. Az $e=uv$ élmenti javítás esetén a $f(v)$ értéket a $min\{f(v),f(u)+l(uv)\}$ értékkel helyettesíthetjük. \begin{itemize}
        \item[(1)] Ha l konzervatív akkor tetsz.  (r,l)-f. b. élmenti javítása (r,l)-fb-t ad.
        \item[(2)] Ha az $f(r,l)$ felső becsléshez nincs érdemi élmenti javítás, akkor $f(v)=dist_l(r,v)$ $\forall v \in V$.
        \end{itemize}
        \item \textbf{Az elméleti javítás}

        \textbf{\textcolor{blue}{Def:}} Tfh $f$ egy $(r,l)$-felső becslés és $uv \in E(G)$. Az $f$ \textcolor{red}{$uv$-elméleti javítása} az az $f'$, amire 
        $
            f'(z) = \begin{cases}
                f(z) & z \neq v \\
                min\{f(v), f(u) + l(uv)\} & z = v
            \end{cases}
        $

        \textbf{\textcolor{orange}{Megf:}} Tfh az $l:E(G)\rightarrow \mathbb{R}$ hosszfüggvény konzervatív és $f(r) = 0$. 
        
        Ekkor (1) Az $f (r,l)$-felső becslés élmenti javítása mindig $(r,l)$-felső becslést ad.

        \textbf{\textcolor{green}{Biz:}} Azt kell megmutatni, hogy van  olyan $rv$-út, aminek a hossza legfeljebb $f(u) + l(uv)$. Ha egy legrövidebb $ru$-utat kiegészítünk az $uv$ éllel, akkor olyan $rv$-élsorozatot kapunk, aminek az összhossza $dist_l (r,u) + l(uv) \leq f(u) + l(uv)$. ,,Könnyen" látható, hogy  az élhosszfüggvény konzervativitása miatt ha van $x$ összhosszúságú $rv$-élsorozat, akkor van legfeljebb $x$ összhosszúságú $rv$-út is. Ezek szerint van legfeljebb $f(u) + l(u,v)$ hosszúságú $uv$-út is, azaz az érdemi élmenti javítás után szintén $(r,l)$-felső becslést kapunk.   \textcolor{blue}{$\Box$} 

        (2) $f (r,l)$-felső becslés (pontosan) $\Longleftrightarrow$ ($f$-en $\nexists$ érdemi élmenti javítás).

        \textcolor{green}{\textbf{Biz:}}  $\Rightarrow$: Ha $f$ pontos, akkor biztosan nincs rajta érdemi élmenti javítás: ha volna, akkor egy felső becslés a pontos érték alá csökkenne, így az élmenti javítás nem  $(r,l)$-felső becslést eredményezne. $\Leftarrow$: Legyen $v \in V(G)$ tetsz, és legyen $P$ egy legrövidebbb $rv$-út. A $P$ egyik éle mentén sincs érdemi élmenti javítás, ezért $P$ minden $u$ csúcsára pontos a felső becslés: $f(u) = dist_l(r,u)$. Ez igaz az út utolsó csúcsára, a tetszőlegesen választott $v$-re is.  \textcolor{blue}{$\Box$}
        
        \item {Dijkstra algoritmus működése:} \begin{itemize}
            \item \textbf{Input}: $G=(V,E)$ irányított gráf, $l:E\rightarrow \mathbb{R}^+$ nemnegatív hosszfüggvény, $r \in V$ gyökér
            \item \textbf{Output}: $dist_l(r,v)$ minden $v \in V$-re.
            \item \textbf{Működés}: Kezdetben $U_0 = \emptyset, f(r) = 0$ és $f(v) = \infty$, ha $v \neq r$. 
            
            Az algoritmus $i$-dik fázisában $(i=1,2,..., |v|)$ a következő történik. \begin{itemize}
                \item[1.] Legyen $u_i$ az $v$ csúcs a $v\backslash u_{i-1}$ halmazból, amelyre $f(r)$ minimális és legyen $u_{i-1} \cup {u_i}$.
                \item[2.] Végezzünk élmenti javításokat minden $u_i$-ből kivezető $u_ix$ élen.
            \end{itemize}
            
            Az output a $|v|$-dik fázik utáni $f$ függvény. Szokás  megjelölni a végső $f(v)$ értékeket beállító éleket. 
            
            Ha az output az $f(r,l)$-felső becslés, akkor \begin{itemize}
                \item[(1)] $f(u_i) \leq f(u_{i+1}) \; \forall 1 \leq u$-re.
                \item[(2)] $f(u_i) \leq f(u_2) \leq \dots \leq (u_n)$
                \item[(3)] élmentijavítás nem változhat $f$-n.
            \end{itemize}
            \item A Dijkstra-algoritmus helyesen működik, azaz $dist_l(r,v)=f(v) \forall v\in V$ teljesül. Az algoritmus során megjelölt élek egy legrövidebb utak fáját alkotják $G$-ben: az $r$ gyökérből minden $r$-ből elérhető csúcshoz vezet olyan legrövidebb út is ami csak megjelölt éleket tartalmaz. 
            \item A Dijkstra-algoritmus lépésszáma legfeljebb. $konst \cdot (n^2+m)$, ahol $n=|v|$ $m=|E|$.
        \end{itemize}
        \item \textbf{\textcolor{blue}{Dijkstra-algoritmus:}} \underline{Input:} $G = (V,E), l : E \rightarrow \mathbb{R}_+, r \in V$. \underline{Output:} $dist_l(r,v) \forall v \in V$ \underline{Működés:} $U_0 := \emptyset, f_0$ a triviális. $(r,l)$-felső becslés. 
				
            Az $i$-dik fázis:

            1. Legyen $U_i := U_{i-1} \cup \{u_i\}$, ahol $u_i$ olyan csúcs a $V  \setminus  U_{i-1}$ halmazból, amelyre $f_{i-1}(v)$ minimális.

            2. $f_i:f_{i-1}$ élmenti javítása minden $U_i$-ből kivezető $u_ix$ élen. Output: $f_{|V|}$. Megjelöljük a végső $f_{|V|}(V)$ értékeket beállító éleket.
            
            \textcolor{orange}{\textbf{Megf:}} Ha a $v$-be vezet megjelölt él, akkor vezet $r$-ből $v$-be megjelölt éleken út, és ennek hozza megegyezik $f_{|V|} (v)$-vel.

            \textcolor{green}{\textbf{Biz:}} $f_{|V|} (r) = 0$, és a megjelölt élek mentén haladva az $f_{|V|}$ érték az élhosszal növekszik.  \textcolor{blue}{$\Box$}

        \item \textbf{Dijkstra helyessége}
				
        \textcolor{orange}{\textbf{Megf:}} Tfh $u_1, u_2, \dots, u_n$ a $G$ csúcsainak sorrendje a Dijkstra-algoritmus végrehajtása után. 

        (1) Ekkor $f_{|V|}(u_i) \leq f_{|V|}(u_{i+1})$ teljesül $\forall 1 \leq i \leq n$.

        \textcolor{green}{\textbf{Biz:}} Az $i$-dik fázisban $f_i(u_i) \leq f_i(u_{i+1})$ teljesült az $u_i$ választása miatt. Ezek után $f_i(u_i)$ már nem változott: $f_{|V|}(u_i) = f_i(u_i)$. Ugyan $f_i(u_{i+1})$ még csökkenhetett, de csak az $u_iu_{i+1}$ él mentén történt javítás miatt, hiszen az $(i+1)$-dik fázisban $u_{i+1}$ bekerült az $U_i$ halmazba, és a hozzá tartozó $(r,l)$-fb már nem csökken tovább. Ekkor $f_{i+1}(u_{i+1}) = min \{f_i(u_{i+1}),f_i(u_i)+l(u_iu_{i+1})\} \geq f_i(u_i)$, mivel $l(u_iu_{i+1}) > 0$. Ezért $f_{|V|}(u_i) = f_i(u_i) \leq f_{i+1}(u_{i+1}) = f_{|V|}(u_{i+1})$  \textcolor{blue}{$\Box$}

        (2) $f_{|V|}(u_1) \leq f_{|V|}(u_2) \leq \dots \leq f_{|V|}(u_n)$

        (3) A Dijsktra-algoritmus outputjaként kaptt $f_{|V|}$-n élmenti javítás nem tud változtatni.

        \textcolor{green}{\textbf{Biz:}} Tegyük fel, hogy $u_iu_j \in E(G)$ a $G$ egy tetszőleges éle. Ha $i > j$, akkor (2) miatt $f_{|V|}(u_i) \geq f_{|V|}(u_j)$, ezért az $u_iu_j$ mentén történő javítás nem tudja $f_{|V|}(u_j)$-t csökkenteni, hisz $l(u_iu_j)$ pozitív. Ha pedig $i < j$, akkor az $i$-dik fázisban megrörtént az $u_iu_j$ mentén történő javítás, és ezt követően $f(u_i)$  nem váltorott, azaz $f_{|V|}(u_i) = f_i(u_i)$. A másik $(r,l)$-felső becslés pedig csak tovább csökkenhetett a késpbbi émj-ok során $f_{|V|}(u_j) \leq f_i(u_j)$. Ezért az $u_iu_j$ él mentén sem az $i$-dik fázisban, sem később nincs érdemi javítás.  \textcolor{blue}{$\Box$}

        \textcolor{orange}{\textbf{Tétel:}} A Dijsktra-algoritmus helyesen működik, azaz $G$ minden csúcsára igaz, hogy $dist(r,v) = f_{|V|}(v)$.

        \textcolor{green}{\textbf{Biz:}} A Dijsktra-algoritmus az $f_0$ triviális $(r,l)$-felső becslésből indul ki, és élmenti javításokat alkalmaz. Így minden $f_i$ (speciálisan $f_{|V|}$ is) $(r,l)$-felső becslés lesz. A fenti (3)-as megfigyelés miatt $f_{|V|}$-n nem  végezhető érdemi élmenti javítás. Ezért egy korábbi (2)-es megfigyelés miatt $f_{|V|}$ pontos $(r,l)$-felső becslés, azaz $f_{|V|}(v) = dist_l(r,v) \forall v \in V(G)$.  \textcolor{blue}{$\Box$}
        
        \item Könnyű olyan példát találni, ahol a Dijkstra-algoritmus konzervatív hosszüggvény esetén hibás eredményt ad. Azonban konzervatív hosszüffvény esetén is igaz, hogy 

        \begin{itemize}
            \item $(r,l)$-fb élmenti javítása $(r,l)$-fb-t eredményez, ill. 
            \item ha egy $(r,l)$-fb-ben nem végezhető erdemi élmenti javítás, akkor pontos.
        \end{itemize}

        konzervatív hosszfüggvény esetén is hasonló startégiát követünk: Élmenti javításokat végzünk a triviális $(r,l)$-fb-en, míg van érdemi javítás.

        \textcolor{blue}{\textbf{Ford-algoritmus:}} \\ \underline{Input:} $G = (V,E)$ irányított, $l:E \rightarrow \mathbb{R}$ konzervatív hosszfüggvény, $r \in V$ gyökérpont. \\ \underline{Output:} $dist_l(r,l)$ minden $ v \in V$ \\ \underline{Működés:} Legyen $E = \{e_1, e_2,\dots, e_m\}$. Kezdetben legy $f(r)=0$ és $v\neq r$ esetén $f(v)=\infty$, Az $i$-dik fázis $i=1 ,2 ,\dots, n-1$ esetén abból áll, hogy elvégezzük az $e_1, e_2,\dots, e_m$ élek menti javításokat. A végén az OUTPUT: $dist_l(r,v) = f(v)$ minden $v$-re. ($dist_l(r,v) = f_{n-1}(v)\forall v\in V$)

        \textcolor{orange}{\textbf{Állítás:}} Ha $l$ konzervatív, akkor $dist_l(v)\; v \in V$-re.

        \textcolor{green}{\textbf{Biz:}} $f_1(v) = dist_l(r,v)$ ha $\exists \leq 1$-élű legrövidebb $rv$-út. $f_2(v) = dist_l(r,v)$ ha $\exists \leq 2$-élű legrövidebb $rv$-út. $\dots$ $f_{n-1}(v) = dist_l(r,v)$ ha $\exists \leq (n-1)$-élű legrövidebb $rv$-út. Tehát $f_{n-1}(v) = dist_l(r,v) \forall v \in V$.  \textcolor{blue}{$\Box$} 

        \textcolor{orange}{\textbf{Megf:}} Ha $f_i = f_{i-1}$, akkor a Ford-algoritmust az $i$-dik fázis után be lehet fejezni, hisz nincs érdemi élmenti javítás, így $f_{n-1} = f_i$.

        \textcolor{blue}{\textbf{Megj:}} Az $f_{n-1}(v)$-t beállító élek legrövidebb utak fáját alkozják. 

        \textcolor{green}{\textbf{Biz:}} A Dijkstra esethez hasonló. Tetszőleges $v$ csúcsból visszafelé követve a végső értékeket beállító éleket $f_{n-1}(v)$ hosszúságú $rv$-utat találunk.  \textcolor{blue}{$\Box$} 

        \textcolor{blue}{\textbf{"Lépésszámanalízis":}} Ha a $|V(G)| = n$ és $|E(G)| = m$, akkor minden fázisban $\leq m$ élmenti javítás, ami $konst \cdot m$ lépés. Ez összesen $\leq konst \cdot (n-1) \cdot m \leq konst \cdot n^3$ lépés, az algoritmus hatékony.

    \end{itemize}

\end{document}