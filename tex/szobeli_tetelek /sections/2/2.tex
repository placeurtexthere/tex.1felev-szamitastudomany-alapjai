\documentclass[../../szobeli.tex]{subfiles}

\begin{document}

\begin{center}
    \noindent\fbox{%
    	\parbox{160mm}{
			\underline{\textbf{Élhozzáadási lemma}} erdő, \textbf{fa}, fák egyszerűbb tulajdonságai: \underline{két levél}, \underline{erdők élszáma}. \textbf{Feszítőfa} \underline{létezése}, feszítőfához tartozó alapkörök és alap vágások.
        }
    }
\end{center}

    2.
    !KÉPLET!
    *Feszítőfa: F a G gráf feszítőfája (ffa), ha F egy G-ből éltörésekkel kapható fa. ha G-nek van feszítőfája <-> összefüggő.
    * Alapvágás, Alapkör: A G gráf F feszítőfájának f éléhez tartozó alap vágást G azon élei alkotják, amik az F-f két komponense között futnak. Az !KÉPLET!  éléhez tartozó alapkör pedig az !KÉPLET!  köre. Megf.: Tegyük fel hogy !KÉPLET!  Ekkor !KÉPLET!  <-> f benne e alapkörében <-> e benne van f alap vágósávban.
    *Minimális költségű ffa: olyan !KÉPLET!  élhalmaz, amire (V, F) fA, É Sk (F) minimális.
    * Kruskal (mohó) algoritmus: !KÉPLET! ahol !KÉPLET!  1. Élek költség szerinti sorba rendezése 2. Döntés az egyes élekről a fenti sorrendben. - Legyen !KÉPLET!  egy gráf, és k: E->R egy tetsz költségfüggvény, (V,F) pedig a G egy ffa F pontosan akkor mkffa, ha minden C-re teljesül, F tartalmazza a G legfeljebb c költségű !KÉPLET! gráf egy feszítő erdejét. - Az algoritmus által kiszámított F élhalmaz a G egy min költségű feszítő erdeje.
    Mkffák struktúrája. G=(V, E)gráf és !KÉPLET!  esetén legyen !KÉPLET!  legfeljebb c költségű élek alkotta feszítő részgráfja G-nek: !KÉPLET! A G gráfon folytatott Kruskal- algoritmus outpontja tartalmazza !KÉPLET!  egy feszítő erdejét minden !KÉPLET!  esetén.
    * Villamos hálózathoz tartozó normál fa keresése: normál fa: G olyan feszítőfája, ami minden feszültségforrást tartalmaz, de egyetlen áramforrást sem (és mindemellett a legtöbb kapacitást és a lehető legkevesebb induktivitást tartalmazza). Normál fa keresése. fesz.forrás (1.) kapacitás (2.) ellenállás (3.) induktivitás (4.) , áramforrás (5.) élköltségekhez keressünk mkffát!Ha ez tartalmaz áramforrást, vagy nem tartalmaz minden feszültségforrást, akkor nincs normál fa, egyébként a mkffa egy normál fa és egyértelmű a megoldás "értelmes" a hálózat.


    \begin{itemize}
        \item \textbf{Két levél:} Legyen $F$ egy tetszőleges fa $n$ csúcson. Ekkor ha $n \geq 2$, akkor $F$-nek legalább két levele van.
        \begin{itemize}
            \item \textcolor{green}{\textbf{Biz:}}: (Algebrai út) A KFL miatt $\sum_{v\in V(G)}(d(v)-2)=\sum_{v\in V(G)}d(v)-2n=2(n-1)-2n=-2$. $F$ minden $v$ csúcsára $d(v) \geq 1$ teljesül, ezért $d(v) - 2 \geq -1$. A fenti összeg csak úgy lehet $-2$, ha $F$-nek legalább 2 levele van.
            \item \textcolor{green}{\textbf{Biz:}}: (Kombinatorikus út) Induljunk el $F$ egy tetszőleges $v$ csúcsából egy sétán, és haladjunk, amíg tununk. Ha sosem akadunk el, akkor előbb-utóbb ismétlődik egy csúcs, és kört találunk. Ezért elakadunk, és az csakis egy $v$-től különböző $u$ levélben történhet. Ha $d(v)=1$, akkor $v$ egy $u$-tól különböző levél. Ha $d(v) \geq 2$, akkor sétát indulhatjuk $v$-ből egy másik él mentén. Ekkor egy $u$-tól különböző levélben akadunk el.
        \end{itemize}
    \end{itemize}

\end{document}