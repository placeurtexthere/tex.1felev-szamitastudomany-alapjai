\documentclass[../../szobeli.tex]{subfiles}

\begin{document}

\begin{center}
    \noindent\fbox{%
    	\parbox{160mm}{
			\underline{\textbf{Élhozzáadási lemma}} erdő, \textbf{fa}, fák egyszerűbb tulajdonságai: \underline{két levél}, \underline{erdők élszáma}. \textbf{Feszítőfa} \underline{létezése}, feszítőfához tartozó alapkörök és alap vágások.
        }
    }
\end{center}
    \begin{itemize}
        \item \underline{\textbf{Élhozzáadási lemma (ÉHL):}} Legyen $G$ irányítatlan gráf és $G' = G + e$. Ekkor az alábbi két esetből pontosan egy valósul meg. \begin{itemize}
            \item[(1)] $G$ és $G'$ komponensei megegyeznek, de $G'$-nek több köre van, mint $G$-nek.
            \item[(2)] $G$ és $G'$ körei megegyeznek, de $G'$-nek eggyel keveseb komponense van, mint $G$-nek. 
        \end{itemize}
        \item Erdő: A körmentes irányítatlan gráfot \textcolor{red}{erdőnek} nevezzük. 
        \item \textbf{Fa:} Az öszefüggő, körmentes irányítatlan gráf neve \textcolor{red}{fa}. \begin{itemize}
            \item $G$ erdő $\Longleftrightarrow  G$ minden komponense fa.
            \item $G$ $n$-csúcsú, $k$-komponensű erdő $\Rightarrow |E(G)| = n-k$.
            \item \textbf{Biz:} Építsük fel $G$-t a $\overline{K_n}$ üresgráfból az élek egyenkénti behúzásával. $G$ körmentes, ezért az ÉHL miatt minden lé zöld: behúzásakor 1-gyel csökken a komponensek száma. A $\overline{K_n}$ üresgráfnak $n$ komponense van, $G$-nek pedig $k$. Ezért pontosan $n-k$ zöld élt kellett behúzni $G$ felépítéséhez.
        \end{itemize}
        \item \textbf{Két levél:} Legyen $F$ egy tetszőleges fa $n$ csúcson. Ekkor ha $n \geq 2$, akkor $F$-nek legalább két levele van.
        \begin{itemize}
            \item \textbf{Biz:}: (Algebrai út) A KFL miatt $\sum_{v\in V(G)}(d(v)-2)=\sum_{v\in V(G)}d(v)-2n=2(n-1)-2n=-2$. $F$ minden $v$ csúcsára $d(v) \geq 1$ teljesül, ezért $d(v) - 2 \geq -1$. A fenti összeg csak úgy lehet $-2$, ha $F$-nek legalább 2 levele van.
            \item \textbf{Biz:}: (Kombinatorikus út) Induljunk el $F$ egy tetszőleges $v$ csúcsából egy sétán, és haladjunk, amíg tununk. Ha sosem akadunk el, akkor előbb-utóbb ismétlődik egy csúcs, és kört találunk. Ezért elakadunk, és az csakis egy $v$-től különböző $u$ levélben történhet. Ha $d(v)=1$, akkor $v$ egy $u$-tól különböző levél. Ha $d(v) \geq 2$, akkor sétát indulhatjuk $v$-ből egy másik él mentén. Ekkor egy $u$-tól különböző levélben akadunk el.
        \end{itemize}
        \item \textbf{Feszítőfa}: $F$ a $G$ gráf feszítőfája (ffa), ha $F$ egy $G$-ből éltörésekkel kapható fa. Ha $G$-nek van feszítőfája $\Leftrightarrow$ összefüggő.
        \item Alapvágás, alapkör: A $G$ gráf $F$ feszítőfájának $f$ éléhez tartozó \textcolor{red}{alap vágást} $G$ azon élei alkotják, amik az $F-f$ két komponense között futnak. Az $e \in E(G) \backslash E(F)$ éléhez tarozó \textcolor{red}{alapkör} pedig az $F+e$ köre. \\ \textbf{Megf:} Tfh $f \in F$ és $e \in E(G) \backslash E(F)$. Ekkor ($F-f+e$ ffa) $\Longleftrightarrow$ ($f$ benne van $e$ alapkörében) $\Longleftrightarrow$ ($e$ benne van $f$ alap vágásában).
    \end{itemize}

\end{document}