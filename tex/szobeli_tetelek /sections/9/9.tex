\documentclass[../../szobeli.tex]{subfiles}

\begin{document}

\begin{center}
    \noindent\fbox{%
    	\parbox{160mm}{
			\textbf{Gráfok síkba ill. gömbre rajzolhatósága, tartomány, sztereografikus projekció}, következményei. Az \underline{\textbf{Euler-féle poliédertétel}, duális kézfogáslemma} és következményei: \underline{felső korlátok az élszámra} és a \underline{minimális fokszáma} egyszerű, síkbarajzolható gráfokon.
        }
    }
\end{center}

    \begin{itemize}
        \item \textbf{Gráfok síkba ill. gömbre rajzolhatósága, tartomány, sztereografikus projekció} 

            \textcolor{blue}{\textbf{Def:}} \textcolor{red}{Síkbarajzolt} (\textcolor{red}{síkbarajzolt}) gráf alatt olyan gráfdiagramot értünk, amiven az  élek nem keresztezik egymást.

            A $G$ gráf \textcolor{red}{síkbarajzolható} (\textcolor{red}{síkbarajzolható}), ha van síkbarajzolt diagramja. 

            Síkbarajzolt gráf \textcolor{red}{tartománya} (\textcolor{red}{lapja}): a diagram komplementerének összefüggő tartománya. A nem korlátos rész neve \textcolor{red}{külső tartomány}. 

            \textcolor{blue}{\textbf{Megj:}} \begin{itemize}
                \item[(1)] A fentieket nem csak egyszerű gráfokra definiáltuk.
                \item[(2)] A síkbarajzolt gráf nem csupán egy gráf, hanem egy \textbf{konkrét} diagram. 
                \item[(3)] Ugyanannak a síkbarajzolható gráfnak nagyon sok lényegesen különböző síkbarajzolt diagramja (lerajzolása) lehet.
                \item[(4)] A görbe (tóruszra) rajzolhatóság hasonlóan definiálható.
            \end{itemize}

            \textcolor{orange}{\textbf{Állítás:}} (A $G$ gráf síkbarajzolható) $\Longleftrightarrow$ ($G$ gömbre rajzolható) 

            \textcolor{green}{\textbf{Biz:}} A sztereografikus projekcióban az északi-sarkból történő vetítés kölcsönösen egyértelmű megfeleltetés a sík pontjai és a síkot a déli-sarkon érintő gömbfelszín pontjai (mínusz északi-sark) között. A síkbarajzolt diagram vetülete gömbre rajzolt lesz ($\Rightarrow$ \checkmark), és az \textit{É}-t nem tartalmazó gömbre rajzolt diagram pedig síkbarajzolttá válik. A $\Leftarrow$ irány igazolásához csupán annyi kell, hogy úgy rajzoljuk $G$-t a gömbre, hogy az \textit{É}-n ne menjen át él. 

        \item Következményei

            \textcolor{orange}{\textbf{Köv:}} síkbarajzolt gráf külső tartománynak nincs kitüntetett szerepe.

            \textcolor{green}{\textbf{Biz:}} Bármely lerajzolás "kifordítható": a diagram átrajzolható úgy, hogy a kiválasztott tartomány legyen a külső tartomány.

            \begin{enumerate}
                \item Vetítsük fel a diagramot a gömbre.
                \item Állítsuk az \textit{É}-t a kiválasztott tartománynak megfelelő gömbi tartomány belsejébe.
                \item Vetítsük vissza a gömbre rajzolt gráfot a síkra. 
            \end{enumerate}

        \item Az \underline{\textbf{Euler-féle poliédertétel}, duális kézfogáslemma} és következményei: \underline{felső korlátok az élszámra} és a \underline{minimális fokszáma} egyszerű, síkbarajzolható gráfokon.

            \textcolor{orange}{\textbf{Köv:}} Bármely konvex poliéder élhálója síkbarajzolható gráf.

            \textcolor{green}{\textbf{Biz:}} A $kx$ poliéder belső pontjából az élháló kivetíthető egy, a poliédert tartalmazó gömbre. Így az élhálóból gömbre rajzolt gráf lesz. Láttuk, hogy minden gömbre rajzolható gráf síkbarajzolható. 

            \textcolor{blue}{\textbf{Megj:}} A $kx$ poliéder élgráfjának tartományai a poliéder lapjainak felelnek meg.

            \textcolor{violet}{\textbf{Terminológia:}} síkbarajzolt $G$ gráf esetén \textcolor{red}{$n, e, t$} ill. \textcolor{red}{$k$} jelöli rendre a $G$ csúcsai, élei, tartományai és komponensei számát.

        \item 

            \textcolor{orange}{\textbf{Duális kézfogáslemma (DKFL):}} Ha $G$ síkbarajzolt gráf, akkor $\sum_{i=1}^{t} l_i=2e$ ahol $l_i$ az $i$-dik lapot határoló élek számát jelöli.

            \textcolor{green}{\textbf{Biz:}} Minden él vagy két különböző lapot határol, vagy ugyanazt a lapot 2-szer. Így minden él 2-vel járul a BO-hoz és a JO-hoz is. 

            \textcolor{blue}{\textbf{Megj:}} A DKFL akkor hasznos, ha a síkbarajzolt gráf lapjairól, a KFL pedig akkor, ha a fokszámokról van információnk.

            \textcolor{orange}{\textbf{Fáry-Wagner-tétel:}} Ha $G$ egyszerű síkbarajzolható gráf, akkor olyan síkbarajzolása is van, amiben minden él egyenes szakasz.

            \includegraphics[width=0.2\textwidth]{./img/1.png}

            \textcolor{orange}{\textbf{Tétel:}} Ha $G$ síkbarajzolt gráf, akkor $n+t=e+k+1$. 

            \textcolor{green}{\textbf{Biz:}} Rajzoljuk meg $G$-t az $n$ csúcsból kiindulva, az élek egyenkénti behúzásával. Kezdetben $t=1, e=0$ és $k=n$, így a bizonyítandó összefüggés fennáll. Tegyük fel, hogy már néhány élt berajzoltunk, még mindig fennáll az összefüggés, és egy éppen az $uv$ élt rajzolunk meg. \begin{itemize}
                \item[$\boxed{1.}$] $u$ és $v$ különböző komponenshez tartoznak. Ekkor $k$ értéke 1-gyel csökken, $e$-é pedig 1-gyel nő. Az ÉHL miatt nem keletkezik kör, tehát nem zárunk körül új tartományt, vagyis $t$ nem változik. Az összefüggés fennmarad.
                \item[$\boxed{2.}$] $u$ és $v$ ugyanahhoz a komponenshez tartoznak. Ekkor $k$ nem változik, $e$ viszont 1-gyel nő. Az ÉHL miatt keletkezik kör, tehát kettévágjuk az $uv$ élt tartalmazó korábbi tartományt. Ezért $t$ is 1-gyel nő, az összefüggés ismét fennmarad. 
            \end{itemize}

            \textcolor{orange}{\textbf{Köv:}} \textcolor{orange}{\textbf{(1)}} Ha $G$ síkbarajzolható, akkor $t$ nem függ a síkbarajzolástól.

            \textcolor{green}{\textbf{Biz:}} $t = e + k + 1 - n$, és a JO nem függ a síkbarajzolástól. 

            \textcolor{orange}{\textbf{!!!(2)!!!}} (\textbf{Euler-formula}) Ha $G$ összefüggő síkbarajzolt gráf, akkor $n + t = e + 2$

            \textcolor{green}{\textbf{Biz:}} Mivel $G$ összefüggő, ezért a fenti Tételben $k = 1$. 

            \textcolor{orange}{\textbf{(3)}} Ha $G$ egyszerű, síkbarajzolható és $n \geq 3$, akkor $e \leq 3n - 6$.

            \textcolor{green}{\textbf{Biz:}} Ilyenkor $G$ minden lapját legalább 3 él határolja, így a DKFL miatt $2e = \sum_{i=1}^{t} l_i \geq 3t$. A Tétel alapján $3n + 2e \geq 3n + 3t = 3e + 3k \geq 3e + 3 + 3 = 3e + 6$, amit rendezve $e \leq 3n - 6$ adódik. 

            \textcolor{orange}{\textbf{(4)}} $G$ egyszerű, síkbarajzolható, $C_3$-mentes és $n \geq 3 \Rightarrow e \leq 2n - 4$. 

            \textcolor{green}{\textbf{Biz:}} Ilyenkor $G$ minden lapját legalább 4 él határolja. A DKFL miatt $2e = \sum_{i=1}^{t}l_i \geq 4t$, így $e \geq 2t$. A Tétel miatt $2n + e \geq 2n + 2t = 2e + 2k + 2 \geq 2e + 2 + 2 = 2e + 4$ Ezt rendezve $e \leq 2n - 4$ adódik. 

            \textcolor{orange}{\textbf{(5)}} Ha $G$ egyszerű, síkbarajzolható, akkor $\delta(G) \leq 5$ (azaz $\exists v : d(v) \leq 5$).

            \textcolor{green}{\textbf{Biz:}} A KFL és \textcolor{orange}{\textbf{(3)}} miatt $\sum_{v\in V(G)}d(v) = 2e \leq 6n - 12$. Ezért van olyan csúcs, amire $d(v) \leq \frac{6n-12}{n} < 6$. 

            \textcolor{orange}{\textbf{(6)}} A $K_5$ és $K_{3,3}$ gráfok egyike sem síkbarajzolható.

            \textcolor{green}{\textbf{Biz:}} A $K_5$ gráf egyszerű, de nem teljesül \textcolor{orange}{\textbf{(3)}}, hiszen $|E(K_5)| = \binom{5}{2} = 10 \nleq 9 = 3 \cdot 5 - 6$. Ezért $K_5$ nem síkbarajzolható. A $K_{3,3}$ gráf egyszerű és $C_3$-mentes, de nem teljesül rá \textcolor{orange}{\textbf{(4)}}, u.i. $|E(K_{3,3})| = 9 \nleq 8 = 2 \cdot 6 - 4$. Ezért $K_{3,3}$ nem síkbarajzolható. 

            \textcolor{blue}{\textbf{Megj:}} Könnyen látható, hogy ha $G$ síkbarajzolható, akkor $G + e$ tóruszra rajzolható bármely $e$ él behúzása esetén. Nem nehéz látni, hogy $K_6$ is tóruszra rajzolható. Sőt: még $K_7$ is az, de $K_8$ már nem. 

            \textcolor{blue}{\textbf{Def:}} \textcolor{red}{Élfelosztás}: az élre egy új, másodfokú csúcs ültetése. \textcolor{red}{Élösszehúzás}: az él törlése és két végpontjának azonosítása. \textcolor{red}{Topologikus G} (\textcolor{red}{soros bővítés}): $G$-ből élfelosztásokkal képzett gráf. 

            \textcolor{orange}{\textbf{Megf:}} Az éltörlés, csúcstörlés, élfelosztás, élösszehúzás operációk mindegyike megőrzi a gráf síkbarajzolható tulajdonságát. 

            \textcolor{orange}{\textbf{Köv:}} (1) Top. $K_5$ top. $K_{3,3}$ nem síkbarajzolható. (2) Ha $G$ síkbarajzolható, akkor $G$-nek nincs se topologikus $K_5$, se topologikus $K_{3,3}$ részgráfja.

            \textcolor{orange}{\textbf{Kuratowski tétele:}} ($G$ síkbarajzolható) $\Longleftrightarrow$ ($G$-nek nincs se topologikus $K_5$, se topologikus $K_{3,3}$ részgráfja) 
            
            \textcolor{violet}{\textbf{Példa:}} Petersen-gráf

    \end{itemize}

\end{document}