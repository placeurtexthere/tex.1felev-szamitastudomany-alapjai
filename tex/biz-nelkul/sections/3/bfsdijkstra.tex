\documentclass[../szamtud.tex]{subfiles}

\begin{document}

\subsection{Általános gráfbejárás $\&$ BFS}

			A gráfbejárási algoritmus az inputgráf csúcsait és éleit fedezi fel. Minden csúcs az eléretlen $\rightarrow$ elért $\rightarrow$ befejezett állapotokat veszi fel. A bejárás akkor ér véget, amint minden csúcs befejezetté vált. 

			1. Van elért csúcs. Választunk egyet, mondjuk $u$-t.
				
			(1a) Ha van olyan $uv$ él, amire $v$ eléretlen, akkor $v$ elérté válik.

			(1b) Ha nincs ilyen $uv$ él, akkor $u$ befejezetté válik.

			2. Nincs elért csúcs.

			(2a) Ha van eléretlen $u$ csúcs, akkor $u$-t elértté tesszük.

			(2b) Ha nincs eléretlen csúcs (azaz $\forall$ csúcs fejezett), akkor END.

			\textbf{\textcolor{blue}{Szélességi bejárás (BFS) szabálya:}}

			Az 1. esetben mindeg a legkorábban elért $u$-t választjuk.

			\textbf{Input:} $G = (V,E)$ (ir/ir.tatlan) gráf, ($v \in V$ gyökérpont\footnote{A gyökérben kezdetben elért állapotú, ezért kivétel az általános szabály alól.}).

			\textbf{Output:} (1) A csúcsok elérési és befejezési sorrendje. (2) Az élek osztályozása:

			\textbf{\textcolor{green}{faél:}} Olyan él, ami mentén egy csúcs elértté vált.

			$uv$ \textbf{\textcolor{brown}{előreél:}} nem faél, de $u$-ból $v$-be faélekből irányított út vezet.

			$uv$ \textbf{\textcolor{blue}{visszaél:}} $v$-ből $u$-ba faélekből irányított út vezet.

			\textbf{\textcolor{red}{keresztél:}} minden más él ($u$ és $v$ közt nincs leszármazott viszony).

			(3) A \textbf{\textcolor{green}{bejárás fája:}} a faélek alkotta részgráf. (A bejárás fája valójában egy gyökereiből kifelé irányított erdő.)

			\textbf{\textcolor{orange}{Megf:}} Irányítatlan esetben az előreél és a visszaél ugyanazt jelenti.

			\textbf{\textcolor{blue}{Terminológia:}} Ha a bejárás fájában $u$-ból $v$-be irányított út vezet, akkor $u$ a $v$ őse és $v$ az $u$ leszármazottja. A faél és az előreél tehát őszből leszármazottba, a visszaél leszármazottból ősbe vezet.

		\subsection{A BFS tulajdonságai}
			
			Nézzük meg egy \textbf{irányított} gráf BFS bejárását is.

			\textbf{\textcolor{Orange}{Állítás:}} Tfh $G=(V,E)$ BFS bejárása után a csúcsok elérési sorrendje $v_1,v_2,\dots,v_n$. Ekkor az alábbiak teljesülnek.

			(1) Ha $i < j$, akkor $v_i$-t hamarabb fejezük be, mint $v_j$-t, továbbá $v_i$ gyerekei megleőzik $v_j$ gyerekeit az elérési sorrendben.

			%\textbf{\textcolor{green}{Biz:}} A $v_i$-t befejezésének pillanatában $v_i$ minden gyereke elért, de $v_j$-nek még egy gyereke sem az. Ezért $v_j$ gyerekeit a $v_i$ csúcs befejezése után érjük el, majd ezt követően fejezzük be $v_j$-t.  \textcolor{blue}{$\Box$} 

			(2) \textbf{Az elérési és befejezési sorrend (BFS esetén) megegyezik.}

			%\textbf{\textcolor{green}{Biz:}} Ha $v_i$-t korábban érjük el, mint $v_j$-t, akkor (1) miatt $v_i$-t korábban is fejezzük be $v_j$-nél. Ezért bármely két csúcs sorrendje ugyanaz az elérési sorrendben mint befejezési sorrendben. Tehát az elérési sorrendnek meg kell egyeznie a befejezési sorrenddel.  \textcolor{blue}{$\Box$} 

			(3) \textbf{Gréfél nem ugorhat át falét:} ha $k < i < j \leq l$ és  $v_i v_j$ faél, akkor $v_k v_l$ nem lehet gráfél.

			%\textbf{\textcolor{green}{Biz:}} Ha $v_k v_l \in E(G)$, akkor $v_l$ szülője $v_k$ vagy egy $v_k$-t megelőző csúcs. (1) miatt $v_j$ szülője sem következhet $v_k$ után, vagyis $v_i$ nem lehet $v_j$ szülője.

			(4) \textbf{Nincs előreél.} (Irányítatlan eset: csak faél és keresztél van.) 

			%\textbf{\textcolor{green}{Biz:}} Indirekt: ha $v_i v_j$ előreél lenne, akkor $v_i$-ből $v_j$-be irányított út vezetne a BFS-fában, és $v_i v_j$ ennek a faélekből álló útnak az utolsó élét átugraná.  \textcolor{blue}{$\Box$} 

			(5) Ha a BFS-fában $k$-élű irányított út vezet $u$-ból $v$-be, akkor $G$-ben nincs $k$-nál kevesebb élű $uv$-út.

			%\textbf{\textcolor{green}{Biz:}} Ha lenne a BFS fa-beli útnál kevesebb elű út $G$-ben, akkor lenne olyan gráfél, ami faélt ugrik át.  \textcolor{blue}{$\Box$} 

			(6) \textbf{A BFS-fa egy legrövidebb utak fája:} a BFS-fa $v_1$ gyökeréből bármely $v_i$ csúcsba vezető faút a $G$ egy legkevesebb élű $v_1 v_i$-útja.

		\subsection{Legrövidebb utak}
				
			\textbf{\textcolor{blue}{Def:}} Adott $G$ (ir) gráf és $l : E(G) \rightarrow \mathbb{R}$ hosszfüggvény esetén egy \textcolor{red}{$P$ út hossza} a $P$ éleinek összhossza: $l(P) = \sum_{e\in E(P)} l(e)$.

			Az $u$ és $v$ csúcsok \textcolor{red}{távolsága} a legrövidebb $uv$-út hossza: $dist_l(u,v):=$ min$\{l(P):P \;uv$-út$\}$ $(\nexists uv$-út$\Rightarrow dist_l (u,v)= \infty.)$ Az $l$  hosszfüggvénye \textcolor{red}{nemnegatív}, ha $l(e) \geq 0$ teljesül minden $e$ élre. Az $l$ hosszvüggvény \textcolor{red}{konzervatív}, ha $G$-ben $\nexists$ negatív összhosszú ir. kör.

			\textbf{\textcolor{red}{Cél:}} Legrövidebb út keresése irányított/irányítatlan gráfban.

			\textbf{\textcolor{orange}{Megf:}} Ha $l(e) = 1 $ a $G$ minden $e$ élére, akkor $l(P)$ a $P$ élszáma. Ezért a BFS-fa minden gyökérből elérhető csúcsba tartalmaz egy legrövidebb utat a gyökérből elérhető csúcsba tartalmaz egy legrövidebb utat a gyökérből, azaz a szélességi bejárás tekinthető egy legrövidebb utat kereső algoritmusnak is. 

			\textbf{\textcolor{blue}{Def:}} Adott $G$ (ir) gráf, $l : E(G) \rightarrow \mathbb{R}$ hosszfüggvény és $r \in V(G)$. \textcolor{red}{$(r,l)$-felső becslés} olyan $f: V(G) \rightarrow \mathbb{R}$ függvény, ami felülről becsli minden csúcs $r$-től mért távolságát: $dist_l (r,v) \geq f(v) \forall v \in V(G)$.

			\textcolor{red}{Triviális} $(r,l)$-felső becslés:
			$
				f(v) = \begin{cases}
					0 & v = r \\
					\infty & v \neq r
				\end{cases}
			$

			\textcolor{red}{Pontos} $(r,l)$-felső becslés: $f(v) = dist_l(r,l)\; \forall v \in V(G)$.

		\subsection{Az elméleti javítás}

			\textbf{\textcolor{blue}{Def:}} Tfh $f$ egy $(r,l)$-felső becslés és $uv \in E(G)$. Az $f$ \textcolor{red}{$uv$-elméleti javítása} az az $f'$, amire 
			$
				f'(z) = \begin{cases}
					f(z) & z \neq v \\
					min\{f(v), f(u) + l(uv)\} & z = v
				\end{cases}
			$

			\textbf{\textcolor{orange}{Megf:}} Tfh az $l:E(G)\rightarrow \mathbb{R}$ hosszfüggvény konzervatív és $f(r) = 0$. 
			
			Ekkor (1) Az $f (r,l)$-felső becslés élmenti javítása mindig $(r,l)$-felső becslést ad.

			%\textbf{\textcolor{green}{Biz:}} Azt kell megmutatni, hogy van  olyan $rv$-út, aminek a hossza legfeljebb $f(u) + l(uv)$. Ha egy legrövidebb $ru$-utat kiegészítünk az $uv$ éllel, akkor olyan $rv$-élsorozatot kapunk, aminek az összhossza $dist_l (r,u) + l(uv) \leq f(u) + l(uv)$. ,,Könnyen" látható, hogy  az élhosszfüggvény konzervativitása miatt ha van $x$ összhosszúságú $rv$-élsorozat, akkor van legfeljebb $x$ összhosszúságú $rv$-út is. Ezek szerint van legfeljebb $f(u) + l(u,v)$ hosszúságú $uv$-út is, azaz az érdemi élmenti javítás után szintén $(r,l)$-felső becslést kapunk.   \textcolor{blue}{$\Box$} 

			(2) $f (r,l)$-felső becslés (pontosan) $\Longleftrightarrow$ ($f$-en $\nexists$ érdemi élmenti javítás).

			%\textcolor{green}{\textbf{Biz:}}  $\Rightarrow$: Ha $f$ pontos, akkor biztosan nincs rajta érdemi élmenti javítás: ha volna, akkor egy felső becslés a pontos érték alá csökkenne, így az élmenti javítás nem  $(r,l)$-felső becslést eredményezne. $\Leftarrow$: Legyen $v \in V(G)$ tetsz, és legyen $P$ egy legrövidebbb $rv$-út. A $P$ egyik éle mentén sincs érdemi élmenti javítás, ezért $P$ minden $u$ csúcsára pontos a felső becslés: $f(u) = dist_l(r,u)$. Ez igaz az út utolsó csúcsára, a tetszőlegesen választott $v$-re is.  \textcolor{blue}{$\Box$}

			\textbf{\textcolor{orange}{Köv:}} Adott $G$, konzervatív $l$ és $r \in V(G)$ esetén ha kiindulunk a triviális $(r,l)$-felső becslésből, és addig végzünk émj-kat, amíg lehet, akkor a végén megkapjuk minden csúcs $r$-től való távolságát.

			\textbf{\textcolor{red}{Itt a jegyzet 17. oldaláról az utolsó kettő pont hiányzik, mivel nem tudom, hogy mennyire lényegesek.}}

			\textbf{\textcolor{blue}{Def:}} Tfh $f$ egy $(r,l)$-felső becslés és $uv \in E(G)$. Az $f$ \textcolor{red}{$uv$-élmenti javítása} az az $f'$, amire 
			$
				f'(z) = \begin{cases}
					f(z) & z \neq v \\
					min\{f(v), f(u) + l(uv)\} & z = v
				\end{cases}
			$

			\textbf{\textcolor{orange}{Megf:}} Tfh az $l : E(G) \rightarrow \mathbb{R}$ hosszfüggvény konzervatív és $f(r) = 0$. Ekkor (1) Az $f (f,l)$-felső becslés élmenti javítása mindig $(r,l)$-felső becslést ad. (2) $f(r,l)$-felső becslés (pontosan) $\Leftrightarrow$ ($f$-en $\nexists$ érdemi élmenti javítás).

			\textbf{\textcolor{blue}{Dijkstra-algoritmus:}} \underline{Input:} $G = (V,E), l : E \rightarrow \mathbb{R}_+, r \in V$. \underline{Output:} $dist_l(r,v) \forall v \in V$ \underline{Működés:} $U_0 := \emptyset, f_0$ a triviális. $(r,l)$-felső becslés. 
			
			Az $i$-dik fázis:

			1. Legyen $U_i := U_{i-1} \cup \{u_i\}$, ahol $u_i$ olyan csúcs a $V  \setminus  U_{i-1}$ halmazból, amelyre $f_{i-1}(v)$ minimális.

			2. $f_i:f_{i-1}$ élmenti javítása minden $U_i$-ből kivezető $u_ix$ élen. Output: $f_{|V|}$. Megjelöljük a végső $f_{|V|}(V)$ értékeket beállító éleket.
			
		\subsection{Dijkstra, egy példán}

			\textbf{\textcolor{blue}{Dijkstra-algoritmus:}} \underline{Input:} $G = (V,E), l : E \rightarrow \mathbb{R}_+, r \in V$. \underline{Output:} $dist_l(r,v) \forall v \in V$ \underline{Működés:} $U_0 := \emptyset, f_0$ a triviális. $(r,l)$-felső becslés. 
				
			Az $i$-dik fázis:

			1. Legyen $U_i := U_{i-1} \cup \{u_i\}$, ahol $u_i$ olyan csúcs a $V  \setminus  U_{i-1}$ halmazból, amelyre $f_{i-1}(v)$ minimális.

			2. $f_i:f_{i-1}$ élmenti javítása minden $U_i$-ből kivezető $u_ix$ élen. Output: $f_{|V|}$. Megjelöljük a végső $f_{|V|}(V)$ értékeket beállító éleket.
			
			\textcolor{orange}{\textbf{Megf:}} Ha a $v$-be vezet megjelölt él, akkor vezet $r$-ből $v$-be megjelölt éleken út, és ennek hozza megegyezik $f_{|V|} (v)$-vel.

			%\textcolor{green}{\textbf{Biz:}} $f_{|V|} (r) = 0$, és a megjelölt élek mentén haladva az $f_{|V|}$ érték az élhosszal növekszik.  \textcolor{blue}{$\Box$}

			\textcolor{orange}{\textbf{Köv:}} Ha a Dijsktra-algoritmus helyes, akkor az algoritmus végén a megjelölt élek egy legrövidebb utak fáját alkotják $r$ gyökérrel.

		\subsection{Dijkstra helyessége}
				
			\textcolor{orange}{\textbf{Megf:}} Tfh $u_1, u_2, \dots, u_n$ a $G$ csúcsainak sorrendje a Dijkstra-algoritmus végrehajtása után. 

			(1) Ekkor $f_{|V|}(u_i) \leq f_{|V|}(u_{i+1})$ teljesül $\forall 1 \leq i \leq n$.

			%\textcolor{green}{\textbf{Biz:}} Az $i$-dik fázisban $f_i(u_i) \leq f_i(u_{i+1})$ teljesült az $u_i$ választása miatt. Ezek után $f_i(u_i)$ már nem változott: $f_{|V|}(u_i) = f_i(u_i)$. Ugyan $f_i(u_{i+1})$ még csökkenhetett, de csak az $u_iu_{i+1}$ él mentén történt javítás miatt, hiszen az $(i+1)$-dik fázisban $u_{i+1}$ bekerült az $U_i$ halmazba, és a hozzá tartozó $(r,l)$-fb már nem csökken tovább. Ekkor $f_{i+1}(u_{i+1}) = min \{f_i(u_{i+1}),f_i(u_i)+l(u_iu_{i+1})\} \geq f_i(u_i)$, mivel $l(u_iu_{i+1}) > 0$. Ezért $f_{|V|}(u_i) = f_i(u_i) \leq f_{i+1}(u_{i+1}) = f_{|V|}(u_{i+1})$  \textcolor{blue}{$\Box$}

			(2) $f_{|V|}(u_1) \leq f_{|V|}(u_2) \leq \dots \leq f_{|V|}(u_n)$

			(3) A Dijsktra-algoritmus outputjaként kaptt $f_{|V|}$-n élmenti javítás nem tud változtatni.
 
			%\textcolor{green}{\textbf{Biz:}} Tegyük fel, hogy $u_iu_j \in E(G)$ a $G$ egy tetszőleges éle. Ha $i > j$, akkor (2) miatt $f_{|V|}(u_i) \geq f_{|V|}(u_j)$, ezért az $u_iu_j$ mentén történő javítás nem tudja $f_{|V|}(u_j)$-t csökkenteni, hisz $l(u_iu_j)$ pozitív. Ha pedig $i < j$, akkor az $i$-dik fázisban megrörtént az $u_iu_j$ mentén történő javítás, és ezt követően $f(u_i)$  nem váltorott, azaz $f_{|V|}(u_i) = f_i(u_i)$. A másik $(r,l)$-felső becslés pedig csak tovább csökkenhetett a késpbbi émj-ok során $f_{|V|}(u_j) \leq f_i(u_j)$. Ezért az $u_iu_j$ él mentén sem az $i$-dik fázisban, sem később nincs érdemi javítás.  \textcolor{blue}{$\Box$}

			\textcolor{orange}{\textbf{Tétel:}} A Dijsktra-algoritmus helyesen működik, azaz $G$ minden csúcsára igaz, hogy $dist(r,v) = f_{|V|}(v)$.

			%\textcolor{green}{\textbf{Biz:}} A Dijsktra-algoritmus az $f_0$ triviális $(r,l)$-felső becslésből indul ki, és élmenti javításokat alkalmaz. Így minden $f_i$ (speciálisan $f_{|V|}$ is) $(r,l)$-felső becslés lesz. A fenti (3)-as megfigyelés miatt $f_{|V|}$-n nem  végezhető érdemi élmenti javítás. Ezért egy korábbi (2)-es megfigyelés miatt $f_{|V|}$ pontos $(r,l)$-felső becslés, azaz $f_{|V|}(v) = dist_l(r,v) \forall v \in V(G)$.  \textcolor{blue}{$\Box$}

			\textbf{\textcolor{blue}{,,Lépésszámanalízis":}} Ha a $G$ gráfnak $n$ csúcsa és $m$ éle van, akkor a Dijkstra-algoritmus $n$-szer keresi meg legfeljebb $n$ szám minimumát, ami összességében legfeljebb $konst \cdot n^2$ lépést igényel. Ezen kívül legfeljebb $m$ élmenti javítást véges, ami $konst' \cdot m$ lépés. Összességében tehát legfeljebb $konst'' \cdot (n^2 + m)$ lépésre van szükség, az algoritmus hatékony. 


\end{document}