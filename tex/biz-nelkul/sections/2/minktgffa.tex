\documentclass[../szamtud.tex]{subfiles}

\begin{document}

    \subsection{Alapkörrendszer, alap vágás renszer}

        Adott egy $G$ gráf és $G$-nek egy $F$rögzített feszítőfája. Ekkor $G$ minden éléhez $F$ meghatározza $G$ éleinek egy fontos részhalmazát. Attól függően, hogy az adott él $F$-hez tartozik-e vagy sem, különböző fajta részhalmazról van szó.

        \textcolor{blue}{\textbf{Def:}} A $G$ gráf $F$ feszítőfájának $f$ éléhez tartozó \textcolor{red}{alap vágást} $G$ azon élei alkotják, amik az $F-f$ két komponense között futnak. Az $e \in E(G) \backslash E(F)$ éléhez tarozó \textcolor{red}{alapkör} pedig az $F+e$ köre.

        \textcolor{orange}{\textbf{Megf:}} Tfh $f \in F$ és $e \in E(G) \backslash E(F)$. Ekkor ($F-f+e$ ffa) $\Longleftrightarrow$ ($f$ benne van $e$ alapkörében) $\Longleftrightarrow$ ($e$ benne van $f$ alap vágásában).

        \textcolor{orange}{\textbf{Köv:}} Az $e \in E(G) \backslash E(F)$ alapkörét $e$ mellett azon $F$-beli élek alkotják, amelyek alapvágása $e$-t tartalmazza. Az $f \in F$ alapvágást $f$ mellett a $G$ azon élei alkotják, amelyek alapköre $f$-t tartalmazza.

    \subsection{Minimális költségű feszítőfa}

        \textcolor{blue}{\textbf{Def:}} Adott a $G = (V,E)$ irányítatlan gráf élein a $k:E \rightarrow \mathbb{R}_+$ költségfüggvény. Az $F \subseteq E$ élhalmaz \textcolor{red}{költsége} az $F$-beli élek összköltsége: $k(F) = \sum_{f\in F}k(F)$.

        Az $F \subseteq E$ élhalmaz $G$-ben \textcolor{red}{minimális költségű feszítőfa} (\textcolor{red}{mkffa}), ha 

        (1) $(V,F)$ a $G$ feszítőfája, és
        
        (2) $k(F) \leq k(F')$ teljesül a $G$ bármely $(V,F')$ feszítőfájára.
        
        Az $F \subseteq E$ élhalmaz $G$-ben \textcolor{red}{minimális költségű feszítő erdeje}, ha 

        (1) $(V,F)$ a $G$ feszítő erdeje, és

        (2) $k(F) \leq k(F')$ teljesül a $G$ bármely $(V,F')$ feszítő erdejére.

        \textcolor{red}{\textbf{Cél:}} Hatékony eljárás mkffa keresésére.
        
        \textcolor{purple}{\textbf{Ötlet:}} Keressük a feszítőfát a tanult módon, az élk egyenkénti behúzásával, az ÉHL szerint zöldre színezett élek halmazaként. 

        Zöld él: olyan él, ami nem alkot kört a korábban felépített élekkel.

        \textcolor{green}{\textbf{Mohó stratégia:}} A feszítőfa építésekor költség szerint növekvő sorrendben döntsünk az élekről, hátha mkffát kapunk a végén.

        \textcolor{blue}{\textbf{Kruskal-algoritmus:}} \underline{Input}: $G=(V,E)$ és $k:E \rightarrow \mathbb{R}_+$ költségfüggvény. \underline{Output}: $F \subseteq E$ \underline{Működés}: Tfh $k(e_1) \leq k(e_2) \leq \dots \leq k(e_m)$, ahol $E=\{e_1,e_2,\dots, e_m\}$. Legyen $F_0=0$, és $i=1,2,\dots,m$-re

            \begin{equation*}
                F_i :=\begin{cases}
                    F_{i-1}\cup \{e_i\} & \text{ha $F_{i-1}\cup \{e_i\}$ körmentes.} \\
                    F_{i-1} & \text{ha $F_{i-1}\cup \{e_i\}$ tartalmaz kört. \;\;\; $F:=F_m$}
                \end{cases}
            \end{equation*}

    \subsection{Minimális költségű feszítőfák struktúrája}

        $G=(V,E)$ gráf és $k:E \rightarrow \mathbb{R}_+$ költségfüggvény esetén legyen $G_c$ a legfeljebb $c$ költségű élek alkotta feszítő részgráfja $G$-nak: $G_c = (V,E_c)$, ahol $E_c := \{e \in E : k(e)\leq c\}$.

        \textcolor{orange}{\textbf{Megf:}} A $G$ gráfon futtotott Kruskal-algoritmus outputja tartalmazza $G_c$ egy feszítő erdejét minden $c \geq 0$ esetén.

        %\textcolor{green}{\textbf{Biz:}} A Kruskal-algoritmus a legfeljebb $c$ költségű ($E_c$-beli) éleket hamarabb dolgozza fel, mint a $c$-nél drágábbakat. Ezért $E_c$ összes élének feldolgozása után pontosan azt az állapotot érjük el, mintha a Kruskal-algoritmust a $G_c$ frágon futtattunk volna. Korábban (az ÉHL előtt) láttuk, hogy az utóbbi algoritmus outputja $G_c$ egye feszítő erdeje. \textcolor{blue}{$\Box$} 

        \textcolor{orange}{\textbf{Lemma:}} Tfh $F=\{f_1,f_2,\dots,f_l\}, k(f_1) \leq k(f_2) \leq \dots \leq k(f_l)$ és $F\cap E_c$ a $G_c$ egy feszítő erdeje $\forall c \geq 0$-ra. Tfh $F' = \{f'_1, f'_2,\dots,f'_l\}$ a $G$ egy feszítő erdejének élei, és $k(f'_1) \leq k(f'_2) \leq \dots \leq k(f'_l)$. Ekkor $k(f_i) \leq k(f'_i)$ teljesül $\forall 1 \leq i \leq l$ esetén, így $k(F) \leq k(F')$.
        
        %\textcolor{green}{\textbf{Biz:}} Indirekt: tfh $k(f_i) > k(f'_i) = c$. Ekkor $|E_c \cap F| <i$, így a feltevés miatt $E_c \cap F$a $G_c$ egy $i$-nél kevesebb élű feszítő erdeje. Az $f'_1, f'_2, \dots, f'_i$ élek is mind $E_c$-beliek, és többen vannak az $E_c\cap F$ feszítő erdő élszámánál. Tehát $f'_1, f'_2, \dots, f'_i$ nem lehet körmentes, így $f'_1, f'_2, \dots, f'_l$ sem. Ez ellentmondás. Tehát $k(f_i) \leq k(f'_i) \:\forall i$. Ezért $k(F) = \sum_{i=1}^{l}k(f_i) \leq \sum_{i=1}^{l}k(f'_i)=k(F')$. \textcolor{blue}{$\Box$} 

        \textcolor{orange}{\textbf{Köv:}} (1) A Kruskal-algoritmus outputja a $G$ gráf egy minimális költségű feszítő erdeje. 
        
        %\textcolor{green}{\textbf{Biz:}} Legyen $F$ a Kruskal-algoritmus outputja. A megfigyelés miatt $F \cap E_c$ a $G_c$ feszítő erdeje $\forall c \geq 0$-ra, ezért a Lemma szerint $k(F) \leq k(F')$ teljesül $G$ tetszőleges $F'$ feszítő erdejére. \textcolor{blue}{$\Box$} 

        \textcolor{orange}{\textbf{Köv:}} (2) Az $F'$ élhalmaz pontosan akkor minimális költségű feszítő erdeje $G$-nek, ha $F' \cap E_c$ a $G_c$ egy feszítő erdeje minden $c \leq 0$-ra.

        %\textcolor{green}{\textbf{Biz:}} A Lemma bizonyítja az elégfégességet.

        %\textcolor{green}{\textbf{Biz:}} A szükségességhez tfh $F' \cap E_c$ nem feszítő erdeje $G_c$-nek, és legyen $F$ a Kruskal-algoritmus outputja. Mivel $F \cap E_c$ a $G_c$ feszítő erdeje, ezért $|F \cap E_c| > |F' \cap E_c|$, így $k(f_i) < k(f'_i)$ teljesül legalább egy $i$-re, és minden $j$-re $k(k_j) \leq k(f'_i)$. Innen $k(F) < k(F')$. \textcolor{blue}{$\Box$}

        \textcolor{orange}{\textbf{Köv:}} (3) Ha a $G$ gárf összefüggő, akkor $G$ feszítő erdeje a $G$ feszítő fája, így a Kruskal-algoritmus mkffát talál. A (2) következmény pedig $G$ mkffáit karakterizálja.
    
    \subsection{Az ötödik elem}
    
        Algoritmusok megadásakor öt dologra figyelünk:

        Input \checkmark, Output \checkmark, Működés \checkmark, Helyesség \checkmark, Lépésszám. 

        Utóbbiról nem volt szó a Kruskal-algoritmus esetében. 

        Tfh $n$ ill. $m$ a $G$ csúcsai ill. élei száma. 

        A Kruskal-algoritmus két részből áll: 

        \;\;\;\;\;\textcolor{blue}{1.} Élek költség szerinti sorbarendezése 

        \;\;\;\;\;\textcolor{blue}{2.} Döntés az egyes élekről a fenti sorrendben. 
        
        \textcolor{blue}{1.} $m$ szám sorbarendezéséhez a buborékrendezés legfeljebb $\binom{m}{2}$ összehasonlítást használ.
        
        \textcolor{blue}{1.} $n$ csúcsú $G$ gráf esetén egy élről döntés megoldható $konst \cdot \log_2n$ lépésben az adatstruktúra karbantartásával együtt is. Az összes döntéshez tehát elegendő $konst \cdot n \cdot \log_2n$ lépés. A Kruskal-algoritmus lépésszáma ezért felülről becsülhető $konst \cdot (n+m) \cdot \log_2(n+m)$-mel.
    

\end{document}