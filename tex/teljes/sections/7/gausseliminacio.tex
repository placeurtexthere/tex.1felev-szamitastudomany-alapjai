\documentclass[../szamtud.tex]{subfiles}

\begin{document}
    \textcolor{blue}{\textbf{Def:}} \textcolor{red}{Lineáris egyenlet: } Ismeretlenek konstansszorosainak összeges konstans. 
    
    \textcolor{blue}{\textbf{Def:}} \textcolor{red}{Lineáris egyenletrendszer:} Véges sok lineáris egyenlet. 

    \textcolor{purple}{\textbf{Példa:}} 
    \begin{align*} 
        3x-4z &= 666 \\ 
        33x-y+77z &= 42 \\
        \sqrt[{\sqrt[3]{2}}]{5}y-(\ln(\cos42))\cdot z &= \pi^{e^\pi} 
    \end{align*}

    \subsection{Elemi sorekvivalens átalakítások}

        \textcolor{blue}{\textbf{Def:}} \textcolor{red}{Lineáris egyenletrendszer kibővített együtthatómátrixa:} a sorok az egyenletek, az oszlopok az ismeretlenek ill. az egyenletek jobb oldalainak felelnek meg, az egyes mezőkben pedig a megfelelő együttható ill. jobb oldali konstans áll.

        \textcolor{purple}{\textbf{Példa:}}

        \begin{minipage}{.4\linewidth}
            \begin{align*}
                x_1-3x_3+5x_4 &= -6 \\
                7x_1+2x_2+3x_3 &= 9 \\
                x_2+7x_3-2x_4 &= 11
            \end{align*}
        \end{minipage}%
        \begin{minipage}{.1\linewidth}
            $\longmapsto$
        \end{minipage}%
        \begin{minipage}{.0\linewidth}
            \begin{align*}
                \begin{pmatrix}
                    x_1 & x_2 & x_3 & x_4 & \bigm| &  \\
                    1 & 0 & -3 & 5 & \bigm| & -6 \\
                    7 & 2 & 3 & 0 & \bigm| & 9 \\
                    0 & 1 & 7 & -2 & \bigm| & 11 
                \end{pmatrix}
            \end{align*}
        \end{minipage}%
    
        \textcolor{blue}{\textbf{Megj:}} A kibővített együtthatómátrix a lineáris egyenletrendszer felírásának egy tömör módja: elkerüljük vele a műveleti- és egyenlőségjelekkel piszmogást, mégis teljesen áttekinthető módon tartalmaz minden lényeges információt.

        \textcolor{blue}{\textbf{Def:}} A kibővített egyhómx \textcolor{red}{elemi sorekvivalens átalakítása (ESÁ)}: (1) sorcsere, (2) sor nemnulla konstanssal végig szorzása, (3) az $i$-dik sor helyettesítése az $i$-dik és $j$-dik sorok (koordinátánkánti) összegével.
    
        \textcolor{orange}{\textbf{Állítás:}} ESÁ hatására a megoldások halmaza nem változik.

        \textcolor{green}{\textbf{Biz:}} Minden ESÁ előtti megoldás megoldás marad a ESÁ után is. Minden ESÁ fordítottja megkapható ESÁ-ok egymásutánjaként is. Ezért minden ESÁ utáni megoldás megoldja az ESÁ előtti előtti rendszert is.

    \subsection{(Redukált) lépcsős alak}
        
        \textcolor{blue}{\textbf{Def:}} Az $M$ mátrix \textcolor{red}{lépcsős alakú} (\textcolor{red}{LA}), ha \begin{itemize}
            \item[(1)] minden sor első nemnulla eleme 1-es (ú.n. \textcolor{red}{vezért1-es}, avagy \textcolor{red}{v1})
            \item[(2)] minden v1 feletti sorban van ettől a v1-től balra eső másik v1. 
        \end{itemize}

        Az $M$ mátrix \textcolor{red}{redukált lépcsős alakú} (\textcolor{red}{RLA}), ha \begin{itemize}
            \item[(3)] $M$ LA és (2) $M$-ben minden v1 felett csak nullák állnak.
        \end{itemize}

        \textcolor{violet}{\textbf{Példa:}} LA mátrix \hspace{30mm} RLA mátrix

            $ \begin{pmatrix}
                \textcolor{red}{1} & 3 & -2 & 0 & 1 & 1 & 7 \\
                0 & 0 & \textcolor{red}{1} & 2 & 0 & 5 & -2 \\
                0 & 0 & 0 & 0 & \textcolor{red}{1} & 1 & -1 \\
                0 & 0 & 0 & 0 & 0 & \textcolor{red}{1} & 5 \\
                0 & 0 & 0 & 0 & 0 & 0 & 0 
            \end{pmatrix}  $ ill. 
            $ \begin{pmatrix}
                0 & \textcolor{red}{1} & 0 & -2 & 0 & 1 & 0 \\
                0 & 0 & \textcolor{red}{1} & 2 & 0 & 5 & 0 \\
                0 & 0 & 0 & 0 & \textcolor{red}{1} & 1 & 0 \\
                0 & 0 & 0 & 0 & 0 & 0 & \textcolor{red}{1} \\
                0 & 0 & 0 & 0 & 0 & 0 & 0 
            \end{pmatrix}  $


            \begin{minipage}{.25\linewidth}
                $ \begin{matrix}
                    x_1 & x_2 & x_3 & x_4 & \bigm| &  \\
                    1 & 3 & 0 & -2 & \bigm| & 1 \\
                    0 & 0 & 1 & 5 & \bigm| & 7 
                \end{matrix}  $
            \end{minipage}
            \begin{minipage}{.05\linewidth}
                $\;\mapsto\;$
            \end{minipage}
            \begin{minipage}{.15\linewidth}
                \begin{align*}
                    x_1-3x_2-2x_4 &= 1 \\
                    2x_3+5x_4 &= 7
                \end{align*}
            \end{minipage}
            \begin{minipage}{.05\linewidth}
                $\;\mapsto\;$
            \end{minipage}
            \begin{minipage}{.1\linewidth}
                \begin{align*}
                    x_2,& x_4 \in \mathbb{R} \\
                    x_1 &= 1-3x_2+2x_4 \\
                    x_3 &= 7 - 5x_4
                \end{align*}
            \end{minipage}
            
            \begin{minipage}{.25\linewidth}
                $ \begin{matrix}
                    x_1 & x_2 & x_3 & x_4 & \bigm| &  \\
                    1 & 3 & 0 & -2 & \bigm| & 1 \\
                    0 & 0 & 1 & 5 & \bigm| & 7 \\
                    0 & 0 & 0 & 0 & \bigm| & 1 \\
                    0 & 0 & 0 & 0 & \bigm| & 0 
                \end{matrix}  $
            \end{minipage}
            \begin{minipage}{.05\linewidth}
                $\;\mapsto\;$
            \end{minipage}
            \begin{minipage}{.15\linewidth}
                \begin{align*}
                    x_1+3x_2-2x_4 &= 1 \\
                    x_3+5x_4 &= 7 \\
                    0 &= 1 \\
                    0 &= 0 \\
                \end{align*}
            \end{minipage}
            \begin{minipage}{.05\linewidth}
                $\;\mapsto\;$
            \end{minipage}
            \begin{minipage}{.05\linewidth}
                \Large\Lightning
            \end{minipage}


        \textcolor{blue}{\textbf{Def:}} Kibővített egyhómx \textcolor{red}{tilos sora}: $0 \; \dots \; 0 \bigm|  x$ alakú sor, ha $x \neq 0$.

        \textcolor{blue}{\textbf{Def:}} A RLA kibővített egyhómx v1-hez tartozó változója \textcolor{red}{kötött}, a többi változó (amihez nem tartozik v1) \textcolor{red}{szabad} (vagy \textcolor{red}{szabad paraméter}). 

        \textcolor{orange}{\textbf{Megf:}} Ha a kibővített egyhómx RLA, akkor (1) minden sor vagy a v1-hez tartozó változó értékadása, vagy tilos sor, vagy csupa0 sor. (2) Ha van tilos sor, akkor nincs megoldás. (3) Ha nincs tilos sor, a szabad paraméterek tetszőleges értékadásához egyértelmű megoldás tartozik.

        \textcolor{orange}{\textbf{Megf:}} A lineáris egyenletrendszer megoldása tekinthető úgy, hogy a lineáris egyenletrendszeregy RLA kibővített egyhómx-szal van megadva.

    \subsection{Gauss-elimináció}

        \textcolor{blue}{\textbf{Gauss-elimináció:}} 

        \underline{Input:} $M \in \mathbb{R}^{n \times k}$ mátrix.

        \underline{Output:} Egy $M$-ből ESÁ-okkal kapható $M' \in \mathbb{R}^{n \times k}$ LA mátrix.

        \underline{Működés:} Az algoritmus fázisokból áll. Az $i$-dik fázisban keresünk egy nemnulla elemet az ($i-1$)-dik sor alatt a lehető legkisebb sorszámú oszlopban. Ha nincs ilyen elem, az algoritmus véget ér. Sorcserével ezt a nemnulla elemet az $i$-dik sorba visszük. Az $i$-dik konstanssal szorzásával ezt az elemet v1-sé alakítjuk. Az $i$-dik sor alatti sorokhoz az $i$-dik sor konstansszorosát hozzáadva kinullázuk a kapott v1 alatti elemeket.


        $ \begin{matrix}
            0 & 0 & -1 & 3 & 5 \\
            3 & 6 & -6 & 0 & 9 \\
            2 & 4 & 0 & 1 & -1 
        \end{matrix}  $
        $\;\rightarrow\;$
        $ \begin{matrix}
            3 & 6 & -6 & 0 & 9 \\
            0 & 0 & -1 & 3 & 5 \\
            2 & 4 & 0 & 1 & -1 
        \end{matrix}  $
        $\;\rightarrow\;$
        $ \begin{matrix}
            \textcolor{red}{1} & 2 & -2 & 0 & 3 \\
            0 & 0 & -1 & 3 & 5 \\
            2 & 4 & 0 & 1 & -1 
        \end{matrix}  $
        $\;\rightarrow\;$
        $ \begin{matrix}
            \textcolor{red}{1} & 2 & -2 & 0 & 3 \\
            0 & 0 & -1 & 3 & 5 \\
            0 & 0 & 4 & 1 & -7 
        \end{matrix}  $
        $\;\rightarrow\;$
        $ \begin{matrix}
            \textcolor{red}{1} & 2 & -2 & 0 & 3 \\
            0 & 0 & \textcolor{red}{1} & -3 & -5 \\
            0 & 0 & 4 & 1 & -7 
        \end{matrix}  $
        $\;\rightarrow\;$
        $ \begin{matrix}
            \textcolor{red}{1} & 2 & -2 & 0 & 3 \\
            0 & 0 & \textcolor{red}{1} & -3 & -5 \\
            0 & 0 & 0 & 13 & 13 
        \end{matrix}  $
        $\;\rightarrow\;$
        $ \begin{matrix}
            \textcolor{red}{1} & 2 & -2 & 0 & 3 \\
            0 & 0 & \textcolor{red}{1} & -3 & -5 \\
            0 & 0 & 0 & \textcolor{red}{1} & 1 
        \end{matrix}  $
        $\Large\checkmark$

        \textcolor{blue}{\textbf{Megj:}} (1) A Gauss-elimináció outputja LA. Az RLA-hoz további lépésekre van szükség: minden v1 felett kinullázhatók az elemek, ha a v1 sorának konstansszorosait a v1 feletti sorokhoz adjuk.

        \textcolor{purple}{\textbf{Példa:}}

        \;\;\;\;\;\;\;
        $ \begin{matrix}
            \textcolor{red}{1} & 2 & -2 & 0 & 3 \\
            0 & 0 & \textcolor{red}{1} & -3 & -5 \\
            0 & 0 & 0 & \textcolor{red}{1} & 1 
        \end{matrix}  $ 
        $\;\;\;\;\;\rightarrow\;\;\;\;\;$
        $ \begin{matrix}
            \textcolor{red}{1} & 2 & -2 & 0 & 3 \\
            0 & 0 & \textcolor{red}{1} & 0 & -2 \\
            0 & 0 & 0 & \textcolor{red}{1} & 1 
        \end{matrix}  $
        $\;\;\;\;\;\rightarrow\;\;\;\;\;$
        $ \begin{matrix}
            \textcolor{red}{1} & 2 & 0 & 0 & -1 \\
            0 & 0 & \textcolor{red}{1} & -3 & -2 \\
            0 & 0 & 0 & \textcolor{red}{1} & 1 
        \end{matrix}  $
        $\Large\checkmark$

        (2) Ha csupán LA (RLA) a cél, eltérhetünk a Gauss-eliminációtól, feltéve, hogy ESÁ-okkal dolgozunk.

        \textcolor{purple}{\textbf{Példa:}}

        \;\;\;\;\;\;\;\;\;\;\;\;\;\;\;\;\;\;\;\;\;
        $ \begin{matrix}
            3 & 2 & -2 & 0 & 3 \\
            2 & 1 & 0 & \sqrt{42} & \sqrt[e]{\pi} 
        \end{matrix}  $
        $\;\;\;\;\;\rightarrow\;\;\;\;\;$
        $ \begin{matrix}
            1 & 1 & -2 & -\sqrt{42} & 3-\sqrt[e]{\pi} \\
            2 & 1 & 0 & \sqrt{42} & \sqrt[e]{\pi} 
        \end{matrix}  $
        $\Large\checkmark$

        (3) A Gauss-elimináció megvalósítható rekurzív algoritmusként is. A GE($M$) (az $M$ mátrixot lépcsős alakra hozó eljárás):

        $\boxed{1.}$ Ha $M$ első oszlopa csupa0, akkor $M'$ az első oszlop törlésével keletkező mátrix. Output: GE($M'$) elé írunk egy csupa0 oszlopot.

        $\boxed{2.}$ Ha $M$ első oszlopa tartalmaz nemnulla elemet, sorcserével és az első sor konstanssal szorzásával az első sor első elemét v1-sé tesszük, majd a v1 alatti elemeket ESÁ-okkal kinullázzuk. Legyen $M'$ az első sor és első oszlop törlésével keletkező mátrix. Output: GE($M'$) elé írunk egy csupa0 oszlopot és az így kapott mátrix fölé a korábban törölt első sort.

        (4) Az $M \in \mathbb{R}^{n \times k}$ Gauss-eliminációja során minden fázisában legfeljebb egy sorcserét, legfeljebb $2n$ sorszorzást és legfeljebb $n$ sorösszeadást hajtunk végre. Ezért minden fázis legfeljebb $konst \cdot nk$ lépést igényel, az összlépésszám legfeljebb $konst \cdot n^2k$. Az input $M$ mátrix $n \cdot k$ elemet tartalmaz, az eljárás hatékony.

    \subsection{Lineáris egyenletrendszer megoldásszáma}

        \textcolor{purple}{\textbf{Láttuk:}}
            \begin{enumerate}
                \item Lineáris egyenletrendszer kibővített együtthatómátrixként is megoldható.
                \item ESÁ nem változtat a megoldásokon.
                \item ESÁ-okkal elérhető a RLA.
                \item A RLA-ból azonnal adódik a megoldás. \begin{itemize}
                    \item Ha az utolsó oszlopban van v1, akkor nincs megoldás.
                    \item Ha az utolsó kivételével minden oszlopban van v1, akkor egyetlen megoldás van.
                    \item Ha az utolsón kívül más oszlopban sincs v1, akkor van szabad paraméter, így végtelen sok különböző megoldás van.
                \end{itemize}
            \end{enumerate}

        \textcolor{orange}{\textbf{Köv:}} Ha a lineáris egyenletrendszernek pontosan egy megoldása van, akkor legalább annyi egynelet van, mint ahány ismeretlen.

        \textcolor{green}{\textbf{Biz:}} Az RLA-ra hozás után nincs szabad paraméter, tehát minden változóhoz tartozik v1. Ezért a kibővített együtthatómátrixnak legalább annyi sora van, mint a változók száma.

        \textcolor{blue}{\textbf{Megj:}} A fenti következmény fordított irányban nem igaz, és lényegében nincs más összefüggés az egyértelmű megoldhatóság, az ismeretlenek és egyenletek száma között.

\end{document}