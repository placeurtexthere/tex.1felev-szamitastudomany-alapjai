\documentclass[12pt]{article}
\usepackage{amssymb}
\usepackage{amsmath}
\usepackage{subfiles}
\usepackage[usenames,dvipsnames]{xcolor}
\usepackage{indentfirst}
\usepackage{microtype}
\usepackage{graphicx}
\usepackage{multicol}
\usepackage{color}
\usepackage{fancybox}
\usepackage{marvosym}
\usepackage{tikz}
\usetikzlibrary{arrows}
\usepackage{geometry}
	\geometry{a4paper, total={170mm,257mm}, left=20mm, top=20mm, }

\renewcommand*\contentsname{Tartalomjegyzék}

\AddToHook{cmd/section/before}{\clearpage}

\begin{document}
\begin{titlepage}
	\centering \vfill
	{\textsc{Budapesti Műszaki és Gazdaságtudományi Egyetem} \par} \vspace{7cm}
	{\huge\bfseries A számítástudomány alapjai\par} \vspace{0.5cm}
	{\large \textsc{Összefoglaló jegyzet}\par} \vspace{1.5cm}
	{\Large\itshape Készítette: Illyés Dávid\par} \vfill

	\noindent\fbox{%
    	\parbox{140mm}{
			\color{red}\textbf{Ez  a jegyzet nagyon hasonlóan van struktúrálva az előadás jegyzetekhez és fő célja, hogy olyan módon adja át a "A Programozás Alapjai 1" nevű tárgy anyagát, hogy az teljesen kezdők számára is könnyen megérthető és megtanulható legyen. }
   		}
	}

	\vfill {\large \today\par}
\end{titlepage}
\tableofcontents
\addtocontents{toc}{~\hfill\textbf{Oldal}\par}

	\section{A gráfelmélet alapjai}

		\subfile{sections/1/grafelmeletialapok.tex}

	\section{Minimális költségű feszítőfák}

		\subfile{sections/2/minktgffa.tex}

	\section{Gráfbejárások és legrövidebb utak}

		\subfile{sections/3/bfsdijkstra.tex}

	\section{Legrövidebb utak, DFS, PERT}

		\subfile{sections/4/fordfloyddfs.tex}

	\section{Euler-séták és Hamilton-körök}

		\subfile{sections/5/eulerhamilton.tex}

	\section{Síkgráfok}

		\subfile{sections/6/sikgrafok.tex}

	\section{Lineáris egyenletrendszerek}

		\subfile{sections/7/gausseliminacio.tex}

	\section{Az $\mathbb{R}^n$ tér alaptulajdonságai}

		\subfile{sections/8/linalgalapok.tex}

	\section{Altér bázisa és dimenziója}

	\section{Négyzetes mátrix determinánsa}

	\section{Mátrixműveletek és lineáris leképezések}

	\section{Mátrix rangja és inverze}

	\section{Mátrixegyenletek}


	\noindent \textcolor{green}{\shadowbox{{\textbf{Biz:} \textcolor{black}{}}}}

\end{document}
